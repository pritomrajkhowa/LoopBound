\section{Related Work}
\label{sec:related}

\subsection{Worst-Case Execution Time (WCET) Analysis} 
WCET of the program is the maximum or worst possible execution time it can take to execute~\cite{Wilhelm:2008:WEP:1347375.1347389}. WCET problems are typically defined for real-time programs and systems which need to satisfy stringent constraints for all iterations and recursion.
%The analysis of upper bounds on the execution is important to demonstrate that such real-time satisfaction of these constraints on resources such as the cache and branch speculation. It is not always possible to obtain upper bounds on execution times for programs as it is an undecidable problem. 
Research on WCET analysis has been an actively researched area for two decades. 
Some important early works ~\cite{Puschner:1989:CME:84842.84850},~\cite{Zhang1993} determine WCET of the program source code without considering the timing effects of the underlying micro-architecture. 
The framework for parametric WCET analysis presented in~\cite{DBLP:conf/wcet/Lisper03} derive iteration bounds from symbolic expressions by Instantiating the symbolic expressions with specific inputs. 
Then it determines the WCET of programs. Another interesting work in this direction is ~\cite{4617284} where proposed approach automatically identifies induction variables and recurrence relations of loops using abstract interpretation~\cite{Ammarguellat:1990:ARI:93542.93583}. Closed-form finding mechanism is templates based where precomputed closed form templates are used to find the closed form solution. Then iteration bounds are derived. The approach presented in ~\cite{Prantltubound} is based on data flow analysis and interval based abstract interpretation to find iteration bounds. 
All these approaches for WCET analysis can handle loop with loops with the relatively simple flow and arithmetic.
For more complex loops iteration bounds are supplied manually in the form of auxiliary program annotations. Fully automated approach~\cite{Knoop:2011:SLB:2341512.2341532} overcome drawbacks of the aforementioned approaches where user guidance is important. Although ~\cite{Knoop:2011:SLB:2341512.2341532} can infer iteration bounds for special classes of loops with non-trivial arithmetic and flow. Unlike ~\cite{Knoop:2011:SLB:2341512.2341532}, our proposed approach can handle more complex program such as programs with multipath control flow. 
In our work, we try to address a problem related to predicting the performance of a program. 
We try to achieve that by considering the cost for each instruction of program and primarily determining accurate bound of iterations and recursion.

\subsection{Invariant Generation and Cost Analysis} 
Loop invariants describe logical properties of the loop that hold for every iteration of the loop. 
SPEED~\cite{speed1} instruments counters into the program at different program locations as artificial variables. 
Then it composed those counters using a proof-rule-based algorithm and computes their upper bounds by using abstract interpretation based techniques to derive linear invariant. 
This approach is further extended to derive more complex loop bounds in in~\cite{Gulwani:2010:RP:1806596.1806630}. 
In this work, 
abstract interpretation is used to compute disjunctive invariants and it uses proof-rules using max, sum, and product operations on bound patterns.
As a result, non-linear symbolic bounds of multi-path loops are obtained by deploying product operations on bound patterns in conjunction with SMT reasoning in the theory of linear arithmetic and arrays. 
\yao{Wht is the limitation of Speed?}
 ~\cite{Albert:2011:CUB:1937961.1937986} is  another approach based on abstract interpretation and so-called cost relations. 
 Cost relations extend recurrence relations and can express recurrence relations with non-deterministic behavior which arise from multi-path loops. 
 Iteration bounds of loops are inferred by constructing evaluation trees of cost relations and computing bounds on the height of the trees. 
 For doing so, linear invariants and ranking functions for each tree node are inferred. 
 Another important approach which also uses counter is ~\cite{Knoop:2011:SLB:2341512.2341532}. But it contrasts SPEED~\cite{speed1}, it uses recurrence equations and introduces a counter for each loop path. 
 
 Unlike the aforementioned techniques, we do not use abstract interpretation but deploy a recurrence solving approach to generate bounds on simple loops. 
 Contrarily to~\cite{speed1,Gulwani:2010:RP:1806596.1806630,Albert:2011:CUB:1937961.1937986}, our method is not limited to multi-path loops that can be translated into simple loops by SMT queries over arithmetic.



\subsection{Recurrence solving and Cost Analysis} 
Recurrence solving is also used in~\cite{10.1007/978-3-642-17511-4_7,Henzinger:2008:VVT:1484209.1484240}.
The work presented in ?? derives loop bounds by solving arbitrary C-finite recurrences and deploying quantifier elimination over integers and real closed fields. 
To this end, ~\cite{Henzinger:2008:VVT:1484209.1484240} uses some algebraic algorithms as black-boxes built upon the computer algebra system Mathematica. 

Contrarily to ~\cite{Henzinger:2008:VVT:1484209.1484240}, we only solve C-finite recurrences of order 1, but, unlike ~\cite{Henzinger:2008:VVT:1484209.1484240}, we do not rely on computer algebra systems and handle more complex multi-path loops.
Symbolic loop bounds in~\cite{10.1007/978-3-642-17511-4_7} are inferred over arbitrarily nested loops with polynomial dependencies among loop iteration variables. 
To this end, C-finiteand hypergeometric recurrence solving are used.
Unlike ~\cite{10.1007/978-3-642-17511-4_7}, we only handle C-finite recurrences of order 1.
%Contrarily to ~\cite{10.1007/978-3-642-17511-4_7}, we, however, design flow refinement techniques to make our approach scalable to the WCET analysis of programs.










