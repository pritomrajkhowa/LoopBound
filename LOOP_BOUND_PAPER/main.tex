\documentclass[10pt,conference]{IEEEtran}
\IEEEoverridecommandlockouts
% The preceding line is only needed to identify funding in the first footnote. If that is unneeded, please comment it out.
\usepackage{mdframed}
\usepackage{fancyvrb}
\usepackage{parcolumns}
\usepackage{algorithm2e}
\usepackage{longtable}
\usepackage{qtree}
\usepackage{lipsum}
\usepackage{adjustbox}
\usepackage[makeroom]{cancel}
\usepackage{pbox}
\usepackage{array}
\usepackage{bookmark}
\usepackage[colorinlistoftodos]{todonotes}
\usepackage{stmaryrd} % semantic     
\usepackage{url}
\usepackage{hyperref}
\hypersetup{
	colorlinks=true,
	linkcolor=blue,
	filecolor=magenta,      
	urlcolor=cyan,
}
\usepackage{comment}
%\usepackage[ligature, inference]{semantic}
%from old version
\usepackage{booktabs} % For formal tables
\usepackage{amsmath}
%\usepackage{amsthm}
%\usepackage{amssymb}
\usepackage{amsfonts}
\usepackage{amssymb}
\usepackage{algorithm2e}
\usepackage{multicol}
\usepackage{latexsym}
\usepackage{multirow}
\usepackage{graphicx}
\usepackage{threeparttable}
\usepackage{tikz}
\usepackage{subcaption}
\usepackage{verbatim}
\usepackage{float}
\usepackage{color}
\usepackage{listings} %For code in appendix
\lstset
{ %Formatting for code in appendix
	language=c++,
	basicstyle=\footnotesize,
	numbers=left,
	stepnumber=1,
	showstringspaces=false,
	tabsize=1,
	breaklines=true,
	breakatwhitespace=false,
	frame = single, 
}

\usepackage{url}
%\usepackage{hyperref}
\usepackage{stmaryrd} % semantic
\usepackage[ligature, inference]{semantic}

\usepackage{flexisym}
\usepackage{mathpartir}
\usepackage{graphicx}
\usepackage{caption}

\newcommand{\SystemName}{{\sc NewName}} %% define the tool name

\def\BibTeX{{\rm B\kern-.05em{\sc i\kern-.025em b}\kern-.08em
    T\kern-.1667em\lower.7ex\hbox{E}\kern-.125emX}}


\newtheorem{defn}{Definition}[section]
\newtheorem{exmp}{Example}[section]
%\newtheorem{definition}{Definition}
%\newtheorem{example}{Example}
\newtheorem{proposition}{Proposition}
\newtheorem{proof}{Proof}
\newcolumntype{L}[1]{>{\raggedright\let\newline\\\arraybackslash\hspace{0pt}}m{#1}}
\newcolumntype{C}[1]{>{\centering\let\newline\\\arraybackslash\hspace{0pt}}m{#1}}
\newcolumntype{R}[1]{>{\raggedleft\let\newline\\\arraybackslash\hspace{0pt}}m{#1}}
\newcommand{\cL}{{\cal L}}
\usepackage{mathtools}
\DeclarePairedDelimiter\ceil{\lceil}{\rceil}
\DeclarePairedDelimiter\floor{\lfloor}{\rfloor}
\begin{document}

\title{Loop Bound Analysis}


\author{\IEEEauthorblockN{Pritom Rajkhowa\IEEEauthorrefmark{1} and Peisen Yao\IEEEauthorrefmark{2} and
		Fangzhen Lin\IEEEauthorrefmark{3}}
	\IEEEauthorblockA{Department of Computer Science\\
		The Hong Kong University of Science and Technology\\
		Clear Water Bay, Kowloon, Hong Kong\\
		Emails: \IEEEauthorrefmark{1}prajkhowa@cse.ust.hk,\IEEEauthorrefmark{2}pyao@cse.ust.hk
		\IEEEauthorrefmark{3}flin@cse.ust.hk}
}

\begin{comment}
\author{\IEEEauthorblockN{1\textsuperscript{st} Given Name Surname}
\IEEEauthorblockA{\textit{dept. name of organization (of Aff.)} \\
\textit{name of organization (of Aff.)}\\
City, Country \\
email address}
\and
\IEEEauthorblockN{2\textsuperscript{nd} Given Name Surname}
\IEEEauthorblockA{\textit{dept. name of organization (of Aff.)} \\
\textit{name of organization (of Aff.)}\\
City, Country \\
email address}
}
\end{comment}

\maketitle

\begin{abstract}
Determining the tight upper bounds on the iteration number of a given program can be used in some critical practical applications. Some prominent applications of the tight upper bounds analysis are program complexity analysis and worst-case execution time (WCET) analysis. This paper describes a system which addresses the problem of automatically inferring iteration bounds of imperative program loops. It makes use of a translation from these programs to first-order logic with
quantifiers on natural numbers. We have implemented our method in a prototype tool \SystemName\ and evaluated it on benchmarks from the literature.

\end{abstract}

\begin{IEEEkeywords}
Automatic Program Verification, First-Order Logic, Mathematical Induction, Recurrences, SMT, Arithmetic
\end{IEEEkeywords}

\section{Introduction}
\label{sec:intro}
%The aim of loop bound analysis is to obtain static information about the upper bound on the number iterations of each loop of a given program during any execution of the program. The loop bound analysis has been widely used in different applications such as resource certification, complexity analysis, and worst case execution time (WCET) analysis.
Static analysis of the program derives the information about the quantitative behavior of programs statically which plays an important role in code optimization and verification. This information can be used to automatically prove certain program properties such as safety property, termination and time complexity of the program. Physical resources are the most important component in the execution of a program. Specially memory-constrained environments such as embedded software systems must satisfy timing-constraints. To ensure the reliability of such programs, it important to prove that these programs terminate on any input within some time and memory limit. For example, stack and heap-space bounds are important
to ensure the reliability of safety-critical systems~\cite{Regehr:2005:ESO:1113830.1113833}. In autonomous systems and application of cloud computing~\cite{Carroll:2010:APC:1855840.1855861,Cohen:2012:ET:2398857.2384676}, it is critical to find and monitor static energy usage information. Worst-case time bounds
can help create constant-time implementations that prevent side-channel attacks~\cite{Barthe:2014:SNC:2660267.2660283,Kasper:2009:FTR:1617722.1617724}. These examples address one of the fundamental questions that quantitative resource information can provide useful feedback for developers.

However, time complexity analysis is useful for any systems working with large data. Consider, for example, sorting algorithms: some of them are practically useless for very large arrays because they could spend hours sorting them, while other finish the computation within seconds. Most of the current static analysis tools in this area compute only termination or an asymptotic complexity of a program. But one can see that there is a significant difference between time complexities $n^2$ and $1000$· $n^2$ , while the asymptotic complexity is the same. We present an algorithm for computing more precise time bounds. The need for the precision can be seen in the embedded systems: hardware which has to perform $n$ operations within a certain time limit is more expensive than the hardware which has to perform just $n$ operations.

Research interest in the field of loop bound analysis is quite recent. The most and development have occurred in the last decade. Previous works such as tools like SPEED~\cite{speed1}, KoAT~\cite{Brockschmidt:2016:ARS:2982214.2866575}, PUBS~\cite{Albert:2012:CAO:2076807.2077025}, Rank~\cite{Alias:2010:MRP:1882094.1882102}, and LOOPUS~\cite{10.1007/978-3-319-08867-9_50} take a
imperative program P and always derive a sound bounds for numerical. Most of these tool lack compositionality. There are another category of tools which are based on the method
of amortized analysis and type systems for functional programs~\cite{Hoffmann:2012:MAR:2362389.2362393,Hofmann:2003:SPH:604131.604148,Hofmann:2006:TAH:2182132.2182135}. The major drawback of these tools is that they
can only associate potential with individual program variables or data structures. There is another categories of tools, such as CAMPY~\cite{Srikanth:2017:CVU:3009837.3009864} and $C^{4}B$~\cite{Carbonneaux:2015:CCR:2737924.2737955} , which provide user interaction to manual proofs of resource bounds. 


In this work, we introduce a new framework for automatically deriving resource bounds on C programs. Another additional feature this framework provides is that when a tool fails to find a resource bound for the input program, then it provides sound user interaction during bound derivation. We have named the framework as \SystemName\ which take a program P from the programmer and translate it to a set of axioms of first-order logic with quantifiers on natural numbers proposed recently by Lin. Then it uses a translated set of axioms derived the resource bound of the input program. When it failed to derive any bound, then it expected bound value $B$ from the programmer. \SystemName\ produces either  of the following output
\begin{itemize}
	\item prove that input bound value $B$ satisfied on all inputs by the program $P$ using translated axioms.
	\item produce a counterexample if $B$ satisfied for some input.
\end{itemize}


Compared to more classical approaches based on ranking functions or amortized reasoning, our tool inherits the benefits of not explicitly generate any invariant.
However, state-of-the-art classical approaches are often unable or inefficient to derive suitable non-linear invariants to find suitable resource bounds. For example, inferring non-linear invariants is a long-standing challenge for conventional safety property verifiers as well~\cite{ball2001automatic,mcmillan2006lazy}. The crux of our technique is to derive non-linear bound resource bounds by constructing a proof that runs of a program path satisfy a given that. \SystemName\'s ability to
reason about non-linear behavior the fact that it uses decision procedure for the
combination of theories of linear arithmetic and uninterpreted functions. We use auxiliary uninterpreted functions to abstract nonlinear functions to build an initial abstraction, then the theorem
prover lazily refines an approximation of the theory of non-linear. If it finds a model, it tests the model as
a model of the path formula under the standard model of non-linear
arithmetic. This place our approach in an advantageous position as compared to other state-of-the-art which failed to find the resource bounds the program with the non-linear body. Another core technical contribution of \SystemName\ is that our technique does not require special or separate analysis for compositional resource
analysis, like exiting approaches~\cite{Carbonneaux:2015:CCR:2737924.2737955}, because of the flexibility provided by our translation. The translated axioms give the relationship between the input and output values of the program variables which provide us the pliability to reason about any property of the program.

In particular, \SystemName\ is integrated with strong recurrence solver which is capable of finding closed forms—effectively of a certain class of P-finite and C-finite recurrences relations as well as certain classes of conditional
recurrences, a simple case of mutual recurrences and multivariable recurrence relations.  \SystemName\ can is able to infer resource bounds on C programs with mutually-recursive functions and nested loops.


However, we want to highlight the major weakness of existing approaches with the help of the following examples of C code snippet

\begin{center}
	\begin{tabular}{ l | c }
		\textbf{$P_1$}&\textbf{$P_2$}\\\hline
		\begin{minipage}{0.2\textwidth}
			\begin{verbatim}
			int x , C;
			while(x < C) 
			{
			   x = x + 1; 
			}
			\end{verbatim}
		\end{minipage} 
	     & 
	\begin{minipage}{0.2\textwidth}
		\begin{verbatim}
	   int x ,y,C;
	   while(x + y < C) 
	   {
	      x = x + 1; 
	      y = y + 1;
	   }
	\end{verbatim}
	\end{minipage}
    
	\end{tabular}
\end{center}

The potential-based techniques and amortized analysis base tool $C^{4}B$ can easily find the bound for the program $P_1$ where $C^{4}B$ return the bound $1.00 |[0, C]|$ for the program $P_1$. But all the tools including $C^{4}B$ failed to find the bound for the program $P_2$. \SystemName\ can find the bound of both programs $ |[0, C-x]|$ and $ |[0, (C-x-y)/2]|$ for $P_1$ and $P_2$ respectively. To illustrate how \SystemName\ works, consider the program $P_2$. Lin's~\cite{Lin20161} translation of the above program $P$ generates the following a set of axioms $\Pi^{\vec{X}}_{P}$ after some simple simplifications where $\vec{X}=\{x,y,C\}$
\begin{eqnarray*}
	&& C_1=C , x_1 = x_3(n), y_1 = y_3(n), \\
	&& x_3(0) = x, y_3(0) = y, \\
	&& x_3(n+1) = x(n)+1, y_3(n+1) = y_3(n)+1, \\
	&&\neg (x_3(N)+y_3(N)<C), \\
	&& \forall n. n<N\rightarrow (x_3(n)+y_3(n)<C)
\end{eqnarray*}

where  $x_1$, $y_1$ and $C_1$ denote the output values of $a$, $b$, $z$, $x$ and $y$, respectively,
$x_6(n)$ and $y_6(n)$ the values of $x$ and $y$ during the $n$th iteration
of the loop, respectively. Also
$N$ is a natural number constant, and the last two axioms say that it
is exactly the number of iterations the loop executes before exiting. Our recurrence solving tool recSolve(RS) can find the closed-form solutions of $x_3(n)$ and $y_3(n)$, which yields the closed-form solution $x_3(n)=n-x$ and $y_3(n)=n-y$ respectively. Those can be used to simplify the recurrence for $y_3(n)$ into following set of axoims after eliminates recurrence relation for $x_3()$, and $y_3()$
\begin{eqnarray*}
	&& x_1 = N+x, y_1 = N+y, C_1=C,\\
	&& (2*N+x+y \geq C) \\
	&& \forall n. n<N\rightarrow (2*n+x+y<C)
\end{eqnarray*}

The system tried to derive using algorithm $XXX$ derived $N=(C-x-y)/2$, then it tried to prove using SMT solver. If it is able to successfully prove that, then it successfully derived bound $ |[0, (C-x-y)/2]|$ .





We have recently constructed a fully automated program verification
system called \SystemName\ \cite{DBLP:conf/synasc/pritom17,viap-array}, for programs with integer assignments and arrays. \SystemName\ is based on the translation Lin \cite{Lin20161} which translates the programs to first-order logic with quantifiers over natural numbers. Translation of the loop body in the program generates a number of axioms based on recursive relationships. Simplify the translated set of axioms by
trying to compute closed-form solutions of recursive relationships. Simplification process help to reduced quantified axioms which help it reducing the complexity in theorem proving. The translation of multi-path programs with loops results in conditional recurrences. Our proposed recurrence solver (RS) which is capable of finding closed-form solutions of conditional recurrences is integrated with \SystemName\ which clearly shows its effectiveness in performance on state-of-art benchmark programs. The theorem proving part uses Z3 directly when the translated axioms have no inductive definitions.
Otherwise, our system searches for a proof by mathematical induction using
Z3 as the base theorem prover. 

The idea of using closed forms of recurrence relations to approximate loops has appeared in a number of previous work. The techniques proposed by Rodr$\acute{i}$guez-Carbonell and Kapur \cite{rodriguez2004automatic} and Kov$\acute{a}$cs\cite{kovacs2004} are based on solving recurrence relations for discovering invariant polynomial equations. On the other hands, works \cite{farzan2015compositional,kincaid2017compositional,kincaid2017non} are based on recurrences analysis which focuses on over-approximate analysis of general loops. Our approach gets the best of both worlds. In contrast, our recurrences precisely
aim to analyze the accurate semantics of general loops.


To summarize, this paper makes the following key contributions:
\vspace{-0.1cm}
\begin{itemize}
	\item We develop the first automatic amortized analysis for C programs. It is naturally compositional, tracks size changes of variables to derive global bounds, can handle mutually-recursive
	functions, generates resource abstractions for functions, derives
	proof certificates, and handles resources that may become available during execution.
	
	\item It can detect logarithmic bounds.
	
	\item It distinguishes different branches inside loops and computes bounds for each of them separately.
	
	\item We prove the soundness of the analysis.
	
	\item We implemented our resource bound analysis in the publicly available tool \SystemName\ .
	\item We present experiments with \SystemName\ on a benchmark of $C^{4}B$ which is more than 2900 lines of C code. A detailed comparison shows that our prototype is the only tool that can derive global bounds for larger C programs while being as powerful as existing tools when deriving linear local bounds for tricky loop and recursion patterns.
\end{itemize}





	
\section{Translation}\label{sec:Translation}
Our translator consider programs in the following language:
\begin{verbatim}
E ::= array(E,...,E) |
operator(E,...,E)
B ::= E = E |
boolean-op(B,...,B)
P ::= array(E,...,E) = E | 
if B then P else P |
P; P |
while B do P 
\end{verbatim}

%Our system supports the full C99 language (according to the standard ISO/IEC 9899).
Given a program $P$, and a language $\vec{X}$, our system generates a set of first-order axioms denoted by $\Pi_P^{\vec{X}}$ that captures the changes of $P$ on $\vec{X}$. Here, a language means a set of functions and predicate symbols. For $\Pi_P^{\vec{X}}$ to be correct, $\vec{X}$ needs to include all program variables in $P$ as well as any functions and predicates that can be changed by $P$. The axioms in the set $\Pi_P^{\vec{X}}$ are generated inductively on the structure of $P$. The algorithm is described in detail in \cite{Lin20161}
and implementation is explained in 
\cite{DBLP:conf/synasc/pritom17}. The inductive cases of translations are given in the table provided in the supplementary information\footnote{\url{https://github.com/VerifierIntegerAssignment/VIAP_ARRAY/blob/master/Document/Inductive_Translation.pdf}}. We have extended our translation programs with arrays; the extension is described in detail in \cite{viap-array}.


The translation for the sequence, conditonal, and while loops remain the same as that of the deterministic programs presented in \cite{Lin20161}.  We have updated those rules by integrating the simplification process, which are briefly described as follows:


\begin{defn} \label{Rule3} When \verb-P- is \verb-if B then P1 else P2- then $\Pi_P^{\vec{X}}$ is the set of following axioms:
	
\vspace{2mm}
$B \rightarrow \varphi$, for each $\varphi \in \Pi_{P1}^{\vec{X}},$

\vspace{2mm}
$\neg B \rightarrow \varphi$, for each $\varphi \in \Pi_{P1}^{\vec{X}},$
\vspace{2mm}

We assume here that for each boolean expression $B$, there is a corresponding formula $B$ in our first-order language.
	

\end{defn}


Translator uses a systematic technique for the simplifying condition of each axoim $\varphi$ $\in$ $\Pi_P^{\vec{X}}$ of the following form where $\Pi_P^{\vec{X}}$ is the set of axioms generated after the translation of the conditional statement:
\begin{eqnarray}
\forall \vec{x}.X_1(\vec{x}) &=& ite(B,\hat{E_{1}},\hat{E}_{2})\label{rec2}
\end{eqnarray}

where $E_{1}$, $E_{2}$ are the translation of the expressions. It uses two different rules which are presented as follows:
\begin{itemize}
	\item When $B$ is boolean expression, and $E_{1}=E_{2}$, then $\varphi$ is simplified to $X_1(\vec{x})=E_{1}$.
	\item When $B$ is boolean expression, and evaluate $B$ is true, then $\varphi$ is simplified to $X_1(\vec{x})=E_{1}$.
\end{itemize}

\begin{defn}\label{Rule4} When \verb-P- is \verb-while B do P1- then $\Pi_P^{\vec{X}}$ is the set of following axioms:
	\vspace{2mm}
	$\varphi(n)$, for each $\varphi \in \Pi_{P1}^{\vec{X}}$,
	\vspace{2mm}
	
	$X^i(\vec{x})$ = $X^i(\vec{x}, 0)$, for each $X^i \in  \vec{X}$
	
	\vspace{2mm}
	$smallest(N, n, \neg B(n)),$
	
	\vspace{2mm}
	$X^i_1(\vec{x})$ = $X^i(\vec{x}, N)$, for each $X^i \in X$
	\vspace{2mm}
	where n is a new natural number variable not already in $\varphi$,
	and $N$ a new constant not already used in $\Pi_{P1}^{\vec{X}}$. 
	For each formula or term $\alpha$, $\alpha(n)$ is defined inductively as
	follows: it is obtained from $\alpha$ by performing the following
	recursive substitutions:
	\begin{itemize}
		\item for each $X^i 
		\in \vec{X}$ , replace all occurrences of 
		$X^{i}_1(e_1, ..., e_k)$ by $X^i(e_1(n), ..., e_k(n)$$, n + 1)$, and
		\item for each program variable $X$ in $\alpha$, replace all occurrences
		of $X(e_1, ..., e_k)$ by $X(e_1(n), ..., e_k(n), n)$. Notice
		that this replacement is for every program variable
		$X$, including those not in $\vec{X}$ .
	\end{itemize}
	$smallest(N, n, \neg B(n))$ is a shorthand for the following formula
	\begin{eqnarray}
	&& \neg  B(N), \label{smallest1}\\
	&&\forall n. n< N\rightarrow B(n) \label{smallest2}
	\end{eqnarray}
\end{defn}


Lets consider $\vec{\sigma}$ represents the set of inductively defined axioms during the translation. If all the condition axioms $\vec{\sigma}$ are of the following form for some $h\geq1$
\begin{eqnarray}
\hspace*{1em}X^i(e_1(n), ..., e_k(n),n+1) &=& \nonumber\\
&&\hspace*{-12em}ite(\theta_1,f_1(X^i(e_1(n), ..., e_k(n)),n),\nonumber\\
&&\hspace*{-10em}ite(\theta_2,f_2(X^i(e_1(n), ...,e_k(n)),n)...,\nonumber\\ &&\hspace*{-8em}ite(\theta_h,f_h(X^i(e_1(n), ..., e_k(n)),n)\nonumber\\
&&\hspace*{-6em},f_{h+1}(X^i(e_1(n), ..., e_k(n)),n))), \nonumber\\
&&\hspace*{-4em} \mbox{ for }1\leq i\leq h \land X^i \in \vec{X}, \label{rec3}
\end{eqnarray}

where $\theta_1,\theta_2,...,\theta_{h}$ are boolean expressions, and 
$f_1(x,y),f_2(x,y),....,f_{h+1}(x,y)$ are polynomial functions of $x$ and $y$. The translator uses a systematic technique to reconstruct $\Pi_P^{\vec{X}}$ as following:

\begin{itemize}
	\item if all the conditions $\theta_i$, $1\leq i\leq h$,
	in it are independent of the recurrence variable $n$, the translator splits the set of axioms $\Pi_P^{\vec{X}}$ to a sequence of set of axioms $\Pi_{P^{1}}^{\vec{X}}$, $\Pi_{P^{2}}^{\vec{X}}$,...,$\Pi_{P^{h}}^{\vec{X}}$, $\Pi_{P^{h+1}}^{\vec{X}}$ corresponding to conditions $\theta_1, \theta_2,...,\theta_h, \theta_{h+1}$ respectively where $\theta_{h+1}   = (\neg \theta_1 \land \neg \theta_2 \land..... \land \neg \theta_h)$.
	For any $\Pi_i$ such that $1\leq i\leq h$. $\Pi_{P^{1}}^{\vec{X}}$ consist of the following set of axioms along with all non-conditional inductively defined axioms of $\Pi_P^{\vec{X}}$  which corresponds to boolean expression $\theta_i$
	
    \vspace{2mm}
	$X^i(\vec{x},0) = X^i(\vec{x})$,  for each $X^i \in  \vec{X}$

	\vspace{2mm}
	
	$smallest(N, n, \neg (B(n) \land \theta_i))$, 
	
	\vspace{2mm}
	$X^i(e_1(n), ..., e_k(n),n+1) = f_i(X^i(e_1(n), ...,e_k(n)),n)$,

	\vspace{2mm} 
	$X^i_1(\vec{x}) = X^i(\vec{x}, N), \text{ for each $X^i$ $\in X$ }$

	\vspace{2mm}

     Then reconstruct $\Pi_P^{\vec{X}}$ from $\Pi_{P^{1}}^{\vec{X}}$, $\Pi_{P^{2}}^{\vec{X}}$,...,$\Pi_{P^{h}}^{\vec{X}}$, ,$\Pi_{P^{h+1}}^{\vec{X}}$ following the translation rule definition presented in~\ref{Rule2}.
     
	\vspace{2mm}
	$\varphi(\vec{X_1}/\vec{Y_1}) \rightarrow \varphi$, for each $\varphi \in \Pi_{P^{1}}^{\vec{X}},$

	\vspace{2mm}
	$\varphi(\vec{X}/\vec{Y_1}) \rightarrow \varphi$, for each $\varphi \in \Pi_{P^{2}}^{\vec{X}}$
	\vspace{2mm}
	
    $\vdots$
    
	\vspace{2mm}
	$\varphi(\vec{X_1}/\vec{Y_h}) \rightarrow \varphi$, for each $\varphi \in \Pi_{P^{h}}^{\vec{X}},$

	\vspace{2mm}
	$\varphi(\vec{X}/\vec{Y_h}) \rightarrow \varphi$, for each $\varphi \in \Pi_{P^{h+1}}^{\vec{X}}$
	\vspace{2mm}
	
	where any $\vec{Y_i} = (Y^1_{i}, ..., Y^k_{i})$ is a tuple of new function symbols such that each $Y^j_{i}$ is of the same arity as $X^j$
in $\vec{X}$ such that $1\leq j \leq k$ , $\varphi(\vec{X_1}/\vec{Y_i} )$ is the result of replacing in $\varphi$
each occurrence of $X^j_1$ by $Y^j_{i}$, and similarly for $\varphi(X/\vec{Y_i})$.

 \item if the condition $\theta_i$, $1\leq i\leq h$, contains only
 arithmetic comparisons $>$ or $<$, polynomial functions of $n$, and does
 not mention $X(n)$. Then translator tries to compute the ranges
 of $\theta_i$ as follows. For each $1\leq i\leq h$, let
 \[
 \phi_i = \neg \theta_1\land\cdots\land\neg \theta_{i-1}\land \theta_i.
 \]
 Thus $\phi_i$ is the condition
 for $X(n+1)$ to take the value $f_i(X(n),n)$. Our system then tries
 to compute $h$ constants $C_1,...C_h$ such that
 \[
 0=C_0<C_1<\cdots<C_h,
 \]
 and for each $i\leq h+1$,
 \vspace{-0.2cm}
 \begin{eqnarray*}
 	&&
 	\forall n. C_{i-1}\leq n< C_i\rightarrow \phi_{\pi(i)}(n) , \\
 	&&\hspace{20mm}\forall n. n\geq C_i\rightarrow \neg \phi_{\pi(i)}(n).
 \end{eqnarray*}
 where $\pi$ is a permutation on $\{1,...,n\}$. The computation of
 these constants and the associated permutation $\pi$ is done using
 an SMT solver: it first computes $k$ for which $\phi_k(0)$ holds, and then
 asks the SMT solver to find a model of
 \begin{eqnarray*}
 	&&
 	\forall n. 0\leq n< C\rightarrow \phi_k(n), \forall n. n\geq C\rightarrow \neg \phi_k(n).
 \end{eqnarray*}
 If it returns a model, then set $C_1=C$ in the model, and let $\pi(1)=k$,
 and the process continues until all $C_i$ are computed. If this fails
 at any point, then the translator aborts this simplification step and return $\Pi_P^{\vec{X}}$ .

Once the system succeeds in computing these constants, the translator reconstruct $\Pi_P^{\vec{X}}$ to a sequence of set of axioms $\Pi_{P^{1}}^{\vec{X}}$, $\Pi_{P^{2}}^{\vec{X}}$,...,$\Pi_{P^{h}}^{\vec{X}}$, ,$\Pi_{P^{h+1}}^{\vec{X}}$ corresponding to constants $C_1, C_2,...,C_h, C_{h+1}$ respectively.
For any $\Pi_i$ such that $1\leq i\leq h$. $\Pi_{P^{1}}^{\vec{X}}$ consist of the following set of axioms along with all non-conditional inductively defined axioms of $\Pi_P^{\vec{X}}$  which corresponds to constant $C_i$

    \vspace{2mm}
$X^i(\vec{x},0) = X^i(\vec{x})$,  for each $X^i \in  \vec{X}$

\vspace{2mm}

$smallest(N, n, \neg (B(n) \land n<C_i))$, 

\vspace{2mm}
$X^i(e_1(n), ..., e_k(n),n+1) = f_i(X^i(e_1(n), ...,e_k(n)),n)$,

\vspace{2mm} 
$X^i_1(\vec{x}) = X^i(\vec{x}, N), \text{ for each $X^i$ $\in X$ }$

\vspace{2mm}

Then reconstruct $\Pi_P^{\vec{X}}$ from $\Pi_{P^{1}}^{\vec{X}}$, $\Pi_{P^{2}}^{\vec{X}}$,...,$\Pi_{P^{h}}^{\vec{X}}$, ,$\Pi_{P^{h+1}}^{\vec{X}}$ following the translation rule definition presented in~\ref{Rule2} as presented in previous case.


	
\end{itemize}





%\input{preliminaries}
\section{Motivating Example Illustrating our Technique}
\label{sec:overview}
In this section, We illustrate the main ideas of our approach using the above simple example. In~\ref{example_details1},
we present an iterative implementation of Naive algorithm for pattern searching. In ~\ref{example_details2}, we presentation translated set axioms generated from the input program by \SystemName\. In~\ref{example_details3}, we give explanation of how \SystemName\ automatically derived expected bound. 










\subsection{An Iterative Implementation Zohar Manna's Version of Integer division algorithm}\label{example1_1}

In this example, we illustrate how \SystemName\ find loop bound where all other states of art tools failed to do. In this example, we present an iterative implementation of Zohar Manna's version of integer division algorithm~\cite{Manna:1974:IMT:542899}.
In~\ref{example1_2}, we present translated set axioms generated from
the input program by translator module of \SystemName\ . In this example, we give an explanation of how our system automatically derives the loop bound of the program.


The C code snippet P with variables $\vec{X}=\{a,b,x,y,z\}$. The two integer variables $a$ and $b$ are assigned with two non-deterministic function call. The algorithm for dividing a number $a$ by another number $b$. When the algorithm terminates, it coming up with a quotient $x$ and a remainder $y$ under the assumption of $a\geq 0$ and $b>0$.


\begin{verbatim}
int a, b, x, y, z;
a = nondetint(); 
b = nondetint();
x = 0; y = 0; z = a;
assume(a >= 0);  
assume(b > 0);
while(z != 0) {
   if (y + 1 == b)  
   { 
      x = x + 1;  
      y = 0;  
      z = z - 1; 
   }
   else 
   { 
      y = y + 1; 
      z = z - 1; 
   }
}
\end{verbatim}

\subsubsection{Translation of Program}\label{example1_2}

The translation module of \SystemName\ implements Lin's~\cite{Lin20161} translation and generates the following  set of axioms $\Pi^{\vec{X}}_{P}$ for program $P$, where 
$\vec{X}=\{a,b,x,y,z\}$.


\begin{eqnarray*}
	&&a_1 = nondetint_2, y_1 = y_6(N), b_1 = nondetint_3,\\ 
	&&x_1 = x_6(N), z_1 = z_6(N),\\
	&&y_6(0)=0, x_6(0)=0, z_6(0)=nondetint_2,\\
	&&\forall n. y_6(n+1) = ite((y_6(n)+1)=nondetint_3\\
	&&\hspace*{5.5em},0,y_6(n)+1),\\
	&&\forall n. x_6(n+1) = ite((y_6(n)+1)=nondetint_3,\\
	&&\hspace*{5.5em}x_6(n)+1,x_6(n)),\\
	&&\forall n. z_6(n+1) = z_6(n)-1,\\
	&&\neg ( z_6(N)\not=0),\\
	&&\forall n. n<N\rightarrow  z_6(n)\not=0
\end{eqnarray*}
where $a_1$, $b_1$, $z_1$, $x_1$ and $y_1$ denote the output values of $a$, $b$, $z$, $x$ and $y$, respectively,
$x_6(n)$, $y_6(n)$ and $z_6(n)$ the values of $x$, $y$ and $z$, respectively during the $n$th iteration
of the loop. The conditional expression $ite(c,e_1,e_2)$ 
has value $e_1$ if $c$ holds and $e_2$ otherwise. Also
$N$ is a natural number constant, and the last two axioms say that it
is exactly the number of iterations the loop executes before exiting.

The translation of assumption results as follows:
\begin{eqnarray}
&&\hspace*{-12.5em}nondetint_2\geq0\label{example1eq2}\\
&&\hspace*{-12.5em}nondetint_3>0 \label{example1eq1} 
\end{eqnarray}


After computing the closed-form solutions
for $x_6()$ and $y_6()$ by RS,  it substituting them in $\Pi^{\vec{X}}_{P}$ and then \SystemName\ eliminates them, and updated $\Pi^{\vec{X}}_{P}$ which is represented by the following axioms:
\begin{eqnarray*}
	&&a_1 = nondetint_2, \\
	&&z_1 = (nondetint_2-N), \\
	&&b_1 = nondetint_3,\\
	&&y_1 = ite(0\leq N \land N<nondetint_3, N,\\ &&ite(N=nondetint_3,0,N-nondetint_3)),\\
	&&x_1 = ite(0\leq N \land N<nondetint_3,0,1),\\
	&&\neg ((nondetint_2-N)\not=0), \\
	&&\forall n. n<N\rightarrow (nondetint_2-n)\not=0
\end{eqnarray*}


\SystemName\ compute the bound $\mathcal{B}^{\vec{X}}_{outer}=N$ by deriving $N_2=n-m+1$ for the following set of equations

\begin{eqnarray*}
	&&\neg ((nondetint_2-N)\not=0), \\
    &&\forall n. n<N\rightarrow (nondetint_2-n)\not=0 
\end{eqnarray*}



\subsection{An Iterative Implementation of Pattern Searching}\label{example_details1}
We illustrate the main ideas of our approach using the another simple example. The C code snippet $P$ with variables $\vec{X}=\{n, m, p, t, r, i, j\}$ where $p$, $t$ and $r$ are two inputs integer array variables of size $m$, $n$ and $n$ respectively such that $n > m$. The code snippet $P$ tries to find the occurs of pattern $p$ in main text array $t$. If $p$ match any substring in $t$, then it stores the starting index of the matched substring of $t$ in $r$.


\begin{verbatim}
1.  int n, m, i, j, k=0;
2.  int p[m], t[n], r[n];
3.  for (i = 0; i <= n - m; i++) { 
4.      for (j = 0; j < m; j++) 
5.          if (t[i + j] != p[j]) 
6.              break; 
7.      if (j == m){ r[k]=i; k=k+1;}
}
\end{verbatim}

The number of comparisons in the worst case is $m*(n-m+1)$. Here we will demonstrate how \SystemName\ automatically derive that. This example to provide an intuitive description of three important aspects of our approach.  Firstly, how the system effectively derive bound without explicitly generating loop invariants. Secondly, how the system effectively handles the program with nested loops. Lastly, how our system handles the program with a unstructured keyword such as break, goto, etc



%\begin{figure}
%	\centering
%	\includegraphics[width=1\textwidth]{figure1.png}%
%	\caption{Motivating loop program from recent works ~\cite{kincaid2017non} and %SV-COMP benchmark~\cite{svcompbenchmark}}%
%	\label{figure1}
%\end{figure}

%\caption{Motivating loop program from recent works ~\cite{kincaid2017non} and SV-COMP benchmark~\cite{svcompbenchmark}}
%This simple example to provide an intuitive description how approach out perform some state of art autmatic verifer.
\subsubsection{Translation of Program}\label{example_details2} 

Our translator would be translated to a set of axioms $\Pi_P^{\vec{X}}$ like the following with resprect to variables $\vec{X}=\{A, B, C, i, j, k, n, d2array\}$:
\begin{eqnarray*}
	&&1. p_1 = p, m_1 = m, t_1 = t, n_1 = n, r1 = r\\
	&&2.\forall x_1,x_2.d1array_1(x_1,x_2) = d1array(x_1,x_2)\\
	&&3.i_1 = i_{14}(N_2)\\
	&&4.j_1 = j_{14}(N_2)\\
	&&5.k_1 = k_{14}(_N2)\\
	&&6.break\_1\_flag_1 = break\_1\_flag_{14}(N_2)\\
	&&7.\forall n_1,n_2.j_5(n_1+1,n_2) =\\ &&\hspace*{2em}ite((ite((d1array(t,(i_{14}(n_2)+j_5(n_1,n_2)))\not=\\
	&&\hspace*{6em}d1array(p,j_5(n_1,n_2))),1,0)=0),\\
	&&\hspace*{9em}(j_5(n_1,n_2)+1),j_5(n_1,n_2))\\
	&&8.\forall n_1,n_2.break\_1\_flag5((n_1+1),n_2) =\\ &&\hspace*{4em}ite((d1array(t,(i_{14}(n_2)+j_5(n_1,n_2)))\not=\\
	&&\hspace*{9em}d1array(p,j_{5}(n_1,n_2))),1,0)\\
	&&9.\forall n_2.j_{5}(0,n_2) = 0\\
	&&10.\forall n_2.break\_1\_flag5(0,n_2) = break\_1\_flag_{14}(n_2)\\
	&&11.\forall n_2.((j_5(N_1(n_2),n_2)\geq m) \lor \\
	&&\hspace*{7em} (break\_1\_flag5(N_1(n_2),n_2)\not=0))\\
	&&12.\forall n_1,n_2.(n_1<N_1(n_2)) \rightarrow ((j5(n_1,n_2)<m) \land\\
	&&\hspace*{9em} (break\_1\_flag5(n_1,n_2)=0))\\
	&&13.\forall n_2.i_{14}(n_2+1) =\\ 
	&&\hspace*{7em}ite((j_5(N_1(n_2),n_2)\not=m),\\
	&&\hspace*{10em}(i_{14}(n_2)+1),i_{14}(n_2))\\
\end{eqnarray*}
\begin{eqnarray*}
	&&14.\forall n_2.j_{14}(n_2+1) = j_5(N_1(n_2),n_2)\\
	&&15.\forall n_2.k_{14}(n_2+1) = ite((j_5(N_1(n_2),n_2)=m),\\
	&&\hspace*{7em}(k_{14}(n_2)+1),k_{14}(n_2))\\
	&&16.d1array_{14}(x_1,x_2,(n_2+1)) = \\
	&&\hspace*{5em}ite((j_5(N_1(n_2),n_2)=m),ite(((x_1=r)\\
	&&\hspace*{7em}\land (x_2=k_{14}(n_2))),(n_2+0),\\
	&&\hspace*{10em}d1array_{14}(x_1,x_2,n_2)),\\
	&&\hspace*{12em}d1array_{14}(x_1,x_2,n_2))\\
	&&17.\forall n_2.break\_1\_flag_{14}((n_2+1))\\
	&&\hspace*{7em} = break\_1\_flag_{5}(N_1(n_2),n_2)\\
	&&18.k_{14}(0) = 0\\
	&&19.j_{14}(0) = j\\
	&&20.break\_1\_flag_{14}(0) = 0\\
	&&21.(N_2>(n-m)) \\
	&&22.(n_2<N_2) \rightarrow (n_2<=(n-m)) 
\end{eqnarray*}


\subsubsection{Bound Analysis}\label{example_details3}

To compute the bound $\mathcal{B}^{\vec{X}}_{Inner}$ using case analysis as system both the equations (11) and (12), which  corresponds to while loop condotion,  involves function(s) generated from program body, the following set of axoims are used 
\begin{eqnarray*}
	&&7.\forall n_1,n_2.j_5(n_1+1,n_2) =\\ &&\hspace*{2em}ite((ite((d1array(t,(i_{14}(n_2)+j_5(n_1,n_2)))\not=\\
	&&\hspace*{6em}d1array(p,j_5(n_1,n_2))),1,0)=0),\\
	&&\hspace*{9em}(j_5(n_1,n_2)+1),j_5(n_1,n_2))\\
	&&8.\forall n_1,n_2.break\_1\_flag5((n_1+1),n_2) =\\ &&\hspace*{4em}ite((d1array(t,(i_{14}(n_2)+j_5(n_1,n_2)))\not=\\
	&&\hspace*{9em}d1array(p,j_{5}(n_1,n_2))),1,0)\\
	&&9.\forall n_2.j_{5}(0,n_2) = 0\\
	&&10.\forall n_2.break\_1\_flag5(0,n_2) = break\_1\_flag_{14}(n_2)\\
	&&11.\forall n_2.((j_5(N_1(n_2),n_2)\geq m) \lor \\
	&&\hspace*{7em} (break\_1\_flag5(N_1(n_2),n_2)\not=0))\\
	&&12.\forall n_1,n_2.(n_1<N_1(n_2)) \rightarrow ((j5(n_1,n_2)<m) \land\\
	&&\hspace*{9em} (break\_1\_flag5(n_1,n_2)=0))
\end{eqnarray*}

After analyzing equations, it derives the following cases.


\textbf{Case 1} When the following condition is holds 

\begin{eqnarray*}
	&&\hspace*{-2em}\forall n_1,n_2.(d1array(t,(i_{14}(n_2)+j_5(n_1,n_2)))\\
	&&\hspace*{6em}\not=d1array(p,j_{5}(n_1,n_2)))\not= False 
\end{eqnarray*}

. Now simplify the equation with respect to the assumption.
\begin{eqnarray*}
	&&7.\forall n_1,n_2.j_5(n_1+1,n_2) = (j_5(n_1,n_2)+1)\\ 
	&&8.\forall n_1,n_2.break\_1\_flag5((n_1+1),n_2) =0\\ 
	&&9.\forall n_2.j_{5}(0,n_2) = 0\\
	&&10.\forall n_2.break\_1\_flag5(0,n_2) = break\_1\_flag_{14}(n_2)
\end{eqnarray*}

Now solve the above equations using RS  with respect to their corresponding initial values which results $j_5(n_1,n_2)=n_1$ and $break\_1\_flag5(n_1,n_2)=0$. After substituting the solution and get rid of equations (7), (8), (9) and (10) with resulted set of axioms are as follows:

\begin{eqnarray*}
	&&11.\forall n_2.(N_1(n_2)\geq m) \lor (0\not=0) \\
	&&12.\forall n_1,n_2.(n_1<N_1(n_2)) \rightarrow (n_1<m) \land (0=0)\\
\end{eqnarray*}

The above set of equations are simplified by getting rid of redundant condition results following a set of equations

\begin{eqnarray*}
	&&\hspace*{-5em}11.\forall n_2.(N_1(n_2)\geq m)  \\
	&&\hspace*{-5em}12.\forall n_1,n_2.(n_1<N_1(n_2)) \rightarrow (n_1<m)\\
\end{eqnarray*}

From the above equations, \SystemName\ derives $N_1(n_2)=m$ which is the bound $\mathcal{B}^{\vec{X}}_{Inner(case_1)}=m$. 

\textbf{Case 2} When the following condition is holds 

\begin{eqnarray*}
	&&\hspace*{-2em}\forall n_1,n_2.(n_1<N_1(n_2) \implies\\ &&\hspace*{1em}(d1array(t,(i_{14}(n_2)+j_5(n_1,n_2)))\\ &&\hspace*{2em}\not=d1array(p,j_{5}(n_1,n_2)))\not= False)\land \\
	&&\hspace*{3em}((d1array(t,(i_{14}(n_2)+j_5(N_1(n_2),n_2)))\\ &&\hspace*{4em}\not=d1array(p,j_{5}(N_1(n_2),n_2)))== False)
\end{eqnarray*}

Now simplify the equation with respect to the assumption.
\begin{eqnarray*}
	&&7.\forall n_1,n_2.j_5(n_1+1,n_2) = ite(0<n_1< N_1(n_2),\\
	&&\hspace*{8em}j_5(n_1,n_2)+1,j_5(n_1,n_2))\\ 
	&&8.\forall n_1,n_2.break\_1\_flag5((n_1+1),n_2) = \\
	&&\hspace*{8em}ite(0<n_1<N_1(n_2),0,1)\\ 
	&&9.\forall n_2.j_{5}(0,n_2) = 0\\
	&&10.\forall n_2.break\_1\_flag5(0,n_2) = break\_1\_flag_{14}(n_2)
\end{eqnarray*}

Now solve the above equations using RS  with respect to their corresponding initial values which results $j_5(n_1,n_2)=ite(0<n_1< N_1(n_2),n_1,N_1(n_2))$ and $break\_1\_flag5(n_1,n_2)=ite(0<n_1< N_1(n_2),0,1)$. After substituting the solution and get rid of equations (7), (8), (9) and (10) with resulted set of axioms are as follows:

\begin{eqnarray*}
	&&11.\forall n_2.(N_1(n_2)\geq m) \lor (1\not=0) \\
	&&12.\forall n_1,n_2.(n_1<N_1(n_2)) \rightarrow (n_1<m) \land (0=0)\\
\end{eqnarray*}

The above set of equations are simplified by getting rid of redundant condition results following a set of equations

\begin{eqnarray*}
	&&\hspace*{-6em}11.\forall n_2.(N_1(n_2)\geq m)  \\
	&&\hspace*{-6em}12.\forall n_1,n_2.(n_1<N_1(n_2)) \rightarrow (n_1<m)\\
\end{eqnarray*}

From the above equations, \SystemName\ derives $N_1(n_2)=m$ which is the bound $\mathcal{B}^{\vec{X}}_{Inner(case_2)}=m$. Now $\mathcal{B}^{\vec{X}}_{Inner}=max(\mathcal{B}^{\vec{X}}_{Inner(case_1)},\mathcal{B}^{\vec{X}}_{Inner(case_2)})=max(m,m)=m$





Similary, \SystemName\ compute the bound $\mathcal{B}^{\vec{X}}_{outer}=n-m+1$ by deriving $N_2=n-m+1$ for the following set of equations

\begin{eqnarray*}
	&&\hspace*{-6em}21.(N_2>(n-m)) \\
	&&\hspace*{-6em}22.(n_2<N_2) \rightarrow (n_2<=(n-m)) 
\end{eqnarray*}









	
\section{Evaluation}
\label{sec:evaluation}

\SystemName\ is implemented  as a stand-alone application, which uses Sympy \cite{joyner2012open} computer algebra system  and Z3 \cite{de2008z3} SMT solver.
The current implementation of \SystemName\ executes using a single thread. The tool along with source code is  publicly available  in the following URL:

\url{https://github.com/VerifierIntegerAssignment/LoopBoundTool}

\SystemName\ is integrated with our recurnce solver framework, RS, to solve a recurrence. All of the experiments are conducted using Intel$\circledR$ Core\texttrademark\ i3 1.8 GHz processor running with 4GB of memory, Ubuntu 16.04 LTS. The source code and the full experiments are available on

\url{https://github.com/pritomrajkhowa/}

\subsection{Benchmarks}
 Our evaluation subjects include a set of C programs gathered 60 challenging loop and recursion patterns from previous publications~\cite{Gulwani:2010:RP:1806596.1806630,speed1,Carbonneaux:2015:CCR:2737924.2737955}. These programs can be divided into three categories
\begin{itemize}
	\item 30 programs are from a benchmark suite of $C^4B$~\cite{Carbonneaux:2015:CCR:2737924.2737955} which is used compared capabitity of $C^4B$ with state-of-art bounds generated tools KoAT~\cite{Brockschmidt:2016:ARS:2982214.2866575}, Rank~\cite{Alias:2010:MRP:1882094.1882102}, LOOPUS~\cite{10.1007/978-3-319-08867-9_50}, SPEED~\cite{speed1} and PUBS~\cite{Albert:2012:CAO:2076807.2077025} in its experiment
	
	\item 20 programs are from the open-source code. Most of the state-of-art bounds generated tools cannot handle these programs. We have chosen these programs to demonstrate the gap exists in current bound generation tools.
	
	\item Another 10 programs are from the open-source code. The main purpose of selection of these program to demonstrate how our tool derive non-linear bound of the programs.  
	

	
	

	
	%\item 15 programs are composed by us. This is class of multi-path (more than two execution path) programs with loops which include some non-linear assertion as well. We composed these program which often contain multiple loops or loop body with multiple conditions.
	
\end{itemize}





\subsection{Experimental setup}


We performed an empirical evaluation in order to answer the following research questions:

\begin{itemize} 
	%\item \textbf{RQ1} What is the primary objective of selection of benchmark programs?
	
	\item \textbf{RQ1} How effectiveness can \SystemName\ derived loop bound of programs compared to the state-of-the-art tools? 
	\item \textbf{RQ2} How our system \SystemName\ tried to fill some of the gap which exits in the curently avaiable state-of-the-art tools?
\end{itemize}



\subsection{Results}


%\subsubsection{\textbf{\textit{RQ1}}}
%The primary objective of the selection of benchmark programs to demostrates that %\SystemName\ can handle program with nondeterministic assignments, non-linear %arithmetic or nested loops which most of the state-of-art failed to 
%from \cite{kincaid2017non} is to demonstrate how \SystemName\ can effectively handle programs with multi-path execution body.

%The columns labeled ``Time'' indicate how long the tool took on each program pair in seconds.  ``TO'' indicates that the tool did not terminate in 500 seconds.

\subsubsection{\textbf{\textit{RQ1}}} 

%To provide a comparison point, we also ran five state-of-the-art fully automated tool for verification or loop-invariant, CPAChecker~\cite{CPAChecker-Tool}, UAutomizer~\cite{UAutomizer-Tool}, ICRA \cite{kincaid2017non}, VeriAbs~\cite{Chimdyalwar:2017:VVA:3080455.3080493}, and Seahorn~\cite{Gurfinkel2015}, generation on the same benchmarks. The main criteria for selecting the system are  
%\vspace{-0.2cm}
%\begin{itemize}
%	\item The system's must be fully automated tool for verification or loop-invariant generation which means a system that verifies a program with only assumption and assertion as specifications given as user input written in C language.
%	\item  Tools must be maintained, so we can consider their latest version.
%\end{itemize}
%\vspace{-0.2cm}
%CPAChecker~\cite{CPAChecker-Tool} and UAutomizer~\cite{UAutomizer-Tool} are the state-ofthe-art program verifiers. We use the versions of CPAChecker and UAutomizer  which used for SV-COMP 2019. ICRA \cite{kincaid2017non} is a program
%verifier which generates invariants based on combines symbolic analysis and abstract interpretation which use recurrence-solving techniques. 

%The results of our evaluation is presented in Table \ref{lbl:evaluation}. For each tool, we record the status and execution time in second. Table \ref{lbl:evaluation} shows that  \SystemName\  were able to prove the partial correctness overwhelming majority of programs to which it was applied. \SystemName\ outperforms competitively compared to the other tools on these benchmarks. It is important state that \SystemName\ have outperformed ICRA \cite{kincaid2017non} in its own set of benchmarks. \SystemName\ takes $53$  seconds on average to prove benchmarks of ICRA \cite{kincaid2017non}. ICRA is the fastest among all the tools we have considered in comparison which took the average $10$ seconds.  The main overhead of \SystemName\  comes from  solving recurrences and in some cases applying proof strategy. 

%\SystemName\ can also handle complex loop program as well as the program of non-linear nature like ICRA
%Similarly, \SystemName\ can also handle complex loop program as well as the program of non-linear nature like ICRA \cite{kincaid2017non}. It also outperform ICRA \cite{kincaid2017non}. On another hand, other tool failed to do so.



\subsubsection{\textbf{\textit{RQ2}}} 
%\vspace{-0.4cm}
%We have observed the integration of RS significantly improved the capacity of our tool \SystemName\ . The results are shown in the 4 and 6 columns of Table \ref{lbl:evaluation} where \SystemName\ without RS can successfully verify the assertion of only 34 programs out of 67. After integration of RS, \SystemName\  can successfully verify the assertion of 54 programs out of 67. Our first version of the system (\SystemName\ 1.0) competed at SV-COMP 2018. The new version (\SystemName\ 1.1) which is presented in this paper that makes use of our newly developed recurrence solver, competed at SV-COMP 2019. As a result, \SystemName\ 1.1(with RS) is able to verify many programs that were out of reach for the older version \SystemName\ 1.0(without RS)\footnote{\url{https://sv-comp.sosy-lab.org/2019/results/results-verified/}}. The main reason is that solving quantified non-linear arithmetics is challenging for current SMT solver. RS can help simplify the formula by geting rid of as many reucnce relations by finding closed form solution which help SMT solver in finding answer.




%\begin{table*}[]
%	\centering
	
%	\caption{Shows the results of the experiments
%		to check assertions.  The two columns under
%		each tool show the running time (in seconds)
%		and the number of proved assertions.}
%	\label{lbl:evaluation}
%	\scalebox{0.9}{
%		\begin{tabular}{|l|l|l|l|l|l|l|l|l|l|l|l|l|l|l|l|}
%			\hline
%			\multirow{2}{*}{\begin{tabular}[c]{@{}l@{}}Benchmark\\ Suite\end{tabular}} & Total & \multicolumn{2}{l|}{ \SystemName} & \multicolumn{2}{l|}{\SystemName+RS} & \multicolumn{2}{l|}{VeriAbs} & \multicolumn{2}{l|}{Sea} & \multicolumn{2}{l|}{ICRA} & \multicolumn{2}{l|}{UAut.} & \multicolumn{2}{l|}{CPA.} \\ \cline{2-16} 
%			& \#A   & Time     & \#A     & Time    & \#A    & Time     & \#A     & Time     & \#A     & Time     & \#A     & Time    & \#A  & Time    & \#A    \\ \hline
%			HOLA  & 46    & 2454     &    20 &  2188.7    &     40    &  1303.6        & 25     &  354   &38     & 235.7    &33       & 1641.9         &20    & 2004.1 & 11       \\ \hline
%			functional  & 21    & 1251     &    14 &  1375    &     14    &  1057        & 5     &  251  &4     & 167.3     &11       & 863         &0    & 833 & 0       \\ \hline
%			Total  & 67    & 3705     &    34 &  3563    &     54    &  2360.6        & 30     &  605   &42     & 399     &44       & 2244.9         &20    & 2837.1 & 11       \\ \hline
%	\end{tabular}}
%\end{table*}




\section{Related Work}
\label{sec:related}

\subsection{Worst-Case Execution Time (WCET) Analysis} 
WCET of the program is the maximum or worst possible execution time it can take to execute~\cite{Wilhelm:2008:WEP:1347375.1347389}. WCET problems are typically defined for real-time programs and systems which need to satisfy stringent constraints for all iterations and recursion.
%The analysis of upper bounds on the execution is important to demonstrate that such real-time satisfaction of these constraints on resources such as the cache and branch speculation. It is not always possible to obtain upper bounds on execution times for programs as it is an undecidable problem. 
Research on WCET analysis has been an actively researched area for two decades. 
Some important early works ~\cite{Puschner:1989:CME:84842.84850},~\cite{Zhang1993} determine WCET of the program source code without considering the timing effects of the underlying micro-architecture. 
The framework for parametric WCET analysis presented in~\cite{DBLP:conf/wcet/Lisper03} derive iteration bounds from symbolic expressions by Instantiating the symbolic expressions with specific inputs. 
Then it determines the WCET of programs. Another interesting work in this direction is ~\cite{4617284} where proposed approach automatically identifies induction variables and recurrence relations of loops using abstract interpretation~\cite{Ammarguellat:1990:ARI:93542.93583}. Closed-form finding mechanism is templates based where precomputed closed form templates are used to find the closed form solution. Then iteration bounds are derived. The approach presented in ~\cite{Prantltubound} is based on data flow analysis and interval based abstract interpretation to find iteration bounds. 
All these approaches for WCET analysis can handle loop with loops with the relatively simple flow and arithmetic.
For more complex loops iteration bounds are supplied manually in the form of auxiliary program annotations. Fully automated approach~\cite{Knoop:2011:SLB:2341512.2341532} overcome drawbacks of the aforementioned approaches where user guidance is important. Although ~\cite{Knoop:2011:SLB:2341512.2341532} can infer iteration bounds for special classes of loops with non-trivial arithmetic and flow. Unlike ~\cite{Knoop:2011:SLB:2341512.2341532}, our proposed approach can handle more complex program such as programs with multipath control flow. I
n our work, we tried addresses a problem related to predicting the performance of a program. We tried to achieve that by considering the cost for each instruction of program and primarily determining accurate bound of iterations and recursion.

\subsection{Invariant Generation and Cost Analysis} 
Loop invariants describe logical properties of the loop that hold for every iteration of the loop. 
SPEED~\cite{speed1} instruments counters into the program at different program locations as artificial variables. Then it composed those counters using a proof-rule-based algorithm and computes their upper bounds by using abstract interpretation based techniques to derive linear invariant. 
This approach is further extended to derive more complex loop bounds in in~\cite{Gulwani:2010:RP:1806596.1806630}. 
In this work, 
abstract interpretation is used to compute disjunctive invariants and it uses proof-rules using max, sum, and product operations on bound patterns.
As a result, non-linear symbolic bounds of multi-path loops are obtained by deploying product operations on bound patterns in conjunction with SMT reasoning in the theory of linear arithmetic and arrays. 
\yao{Wht is the limitation of Speed?}
 ~\cite{Albert:2011:CUB:1937961.1937986} is  another approach based on abstract interpretation and so-called cost relations. 
 Cost relations extend recurrence relations and can express recurrence relations with non-deterministic behavior which arise from multi-path loops. 
 Iteration bounds of loops are inferred by constructing evaluation trees of cost relations and computing bounds on the height of the trees. 
 For doing so, linear invariants and ranking functions for each tree node are inferred. 
 Another important approach which also uses counter is ~\cite{Knoop:2011:SLB:2341512.2341532}. But it contrasts SPEED~\cite{speed1}, it uses recurrence equations and introduces a counter for each loop path. 
 
 Unlike the aforementioned techniques, we do not use abstract interpretation but deploy a recurrence solving approach to generate bounds on simple loops. 
 Contrarily to~\cite{speed1,Gulwani:2010:RP:1806596.1806630,Albert:2011:CUB:1937961.1937986}, our method is not limited to multi-path loops that can be translated into simple loops by SMT queries over arithmetic.



\subsection{Recurrence solving and Cost Analysis} 
Recurrence solving is also used in~\cite{10.1007/978-3-642-17511-4_7,Henzinger:2008:VVT:1484209.1484240}.
The work presented in ?? derives loop bounds by solving arbitrary C-finite recurrences and deploying quantifier elimination over integers and real closed fields. 
To this end, ~\cite{Henzinger:2008:VVT:1484209.1484240} uses some algebraic algorithms as black-boxes built upon the computer algebra system Mathematica. 

Contrarily to ~\cite{Henzinger:2008:VVT:1484209.1484240}, we only solve C-finite recurrences of order 1, but, unlike ~\cite{Henzinger:2008:VVT:1484209.1484240}, we do not rely on computer algebra systems and handle more complex multi-path loops.
Symbolic loop bounds in~\cite{10.1007/978-3-642-17511-4_7} are inferred over arbitrarily nested loops with polynomial dependencies among loop iteration variables. 
To this end, C-finiteand hypergeometric recurrence solving are used.
Unlike ~\cite{10.1007/978-3-642-17511-4_7}, we only handle C-finite recurrences of order 1.
%Contrarily to ~\cite{10.1007/978-3-642-17511-4_7}, we, however, design flow refinement techniques to make our approach scalable to the WCET analysis of programs.











%\section{Conclusion}
\label{sec:conclu}
We have presented an automatic tool for computing closed form solution of recurrences. We have integrated this to our previously developed automated program verifier \SystemName\ . The interaction extended the capability of the tools. We have demonstrated that by proving the correctness of challenging programs
from the benchmark of existing state-of-the-art tools. To the best of our knowledge, this is the first work that can find the closed form solution of conditional recurrence function. In the future, we plan to extend recSolve(RS) to handle a wide range of conditional recurrence, mutual recurrence, and conditional mutual recurrence functions. We attempted to apply it to more applications such as analysis of the worst-case execution time, automatic generation of the unit test case and automatic program repair. For future work, we want to extend our systems, \SystemName\ with a more powerful tactic for wide classes of programs, and for proving properties regarding termination.

\section*{Acknowledgment}

We would like to thank Peisen YAO and Prashant Saikia for useful
discussions. We are grateful to the developers of
Z3 and sympy for making their systems available for open use. All errors remain ours.










 




\bibliographystyle{IEEEtran}
\bibliography{equiv}
%\appendix
%\input{MoreProof}
%\input{MoreData}

%\onecolumn
%% Appendix
\appendix
\section{Appendix}
\label{sec:appendix}

\subsection{SPEED 1}

The following example is taken from benchmark of SPEED~\cite{speed1}.
\begin{verbatim}
1.  int n; x=0; y=0;
2.  while(1){
3.      if (x < n) { 
4.          y=y+1;
5.          x=x+1; 
6.      } else if (y > 0)
7.         y=y-1;
8.      else 
9.         break;
10.    }
11.  }
\end{verbatim}



Our translator would be translated to a set of axioms $\Pi_P^{\vec{X}}$ like the following:
\begin{eqnarray*}
	&&1. n_1 = n, y_1 = y_7(N_1), x_1 = x_7(N_1)\\
	&&2.break\_1\_flag1 = break\_1\_flag7(_N1)\\
	&&3.\forall n_1,y_7(n_1+1) = ite(x_7(n_1)<n),\\ &&\hspace*{2em}ite((y_7(n_1)>0),(y_7(n_1)-1),y_7(n_1)))\\ 
	&&4.\forall n_1,x_7((n_1+1)) = ite((x_7(n_1)<n),\\ &&\hspace*{2em}(x_7(n_1)+1),x_7(n_1))\\ 
	&&5.\forall n_1,break\_1\_flag_7((n_1+1)) = ite((x_7(n_1)<n),\\ &&\hspace*{2em}0,ite((y_7(n_1)>0),0,1))\\
	&&6.y_7(0) = 0, x_7(0) = 0, break\_1\_flag_7(0) = 0\\
	&&7.((1<=0) \lor (break\_1\_flag7(N_1)\not=0))\\
	&&8.\forall n_1.(n_1<N_1) \rightarrow ((1>0)\\
	&&\hspace*{4em} \land (break\_1\_flag_7(n_1)==0))
\end{eqnarray*}



By analyzing the equations (7) and (8), we have found that loop terminates only when break excuted. We need to check all possible exution path of this loop end up excuting break. For that we will use case analysis to we find the closed form solution of the conditional recurrences of $x_7$ and $y_7$

\subsubsection{Case 1} When $x_7(0)<n$ holdes, RS will find the following closed form solution for the conditional recurrences of $x_7$ and $y_7$ 
\begin{eqnarray*}
	&&\hspace*{-2em}x_7(n_1)=ite(0<n_1\leq C_1,n_1,ite(n_1\leq C_2,C_1,C_1))\\
	&&\hspace*{-2em}y_7(n_1)=ite(0<n_1\leq C_1,0-n_1,ite(n_1 \leq C_2,\\
	&&\hspace*{5em}n-C_1,C_2-C_1))
\end{eqnarray*}	
We can derived $C_1=n$ and $C_2=2*n$ by solving following additional axoims
\begin{eqnarray*}
	&&\hspace*{-2em}\forall n_1. 0 \leq n_1<C_1 \rightarrow n_1<n \\
	&&\hspace*{-2em}\neg (C_1 < n)\\
	&&\hspace*{-2em}\forall n_1. C_1 \leq n_1<C_2 \rightarrow \neg (C_1 < n) \land n_1-C_1 > 0\\
	&&\hspace*{-2em}\neg (C_1 < n) \land \neg (C_2-C_1 > 0)
\end{eqnarray*}

From the above equations, We can derives $N_1=2*n$ which is the bound $\mathcal{B}^{\vec{X}}_{(case_1)}=2*n$


\subsubsection{Case 2} When holdes $\neg (x_7(0)<n) \wedge \neg (y_7(0)>0)$ , RS will find the following closed form solution for the conditional recurrences of $x_7$ and $y_7$ along with additional axoims
\begin{eqnarray*}
	&&\hspace*{-2em}x_7(n_1)=0, y_7(n_1)=0
\end{eqnarray*}

From the above equations, We can derives $N_1=1$ which is the bound $\mathcal{B}^{\vec{X}}_{(case_2)}=1$

Another case $\neg (x_7(0)<n) \wedge (y_7(0)>0)$ is invalid. Now $\mathcal{B}^{\vec{X}}=max(\mathcal{B}^{\vec{X}}_{(case_1)},\mathcal{B}^{\vec{X}}_{(case_2)})=max(1,2*n)=2*n$

\subsection{SPEED 2}

The following example is taken from benchmark of SPEED~\cite{speed1}.
\begin{verbatim}
1. int n, m, va=n, vb=0;
2. while (va > 0) {
3.     if (vb < m) { 
4.         vb=vb+1; va=va-1;
5.     } else {
6.          vb=vb-1; vb=0;
7. }
8.}
\end{verbatim}




Our translator would be translated to a set of axioms $\Pi_P^{\vec{X}}$ like the following:
\begin{eqnarray*}
	&&1. n_1 = n, m_1 = m, va_1 = va_5(N_1), vb_1 = vb_5(N_1)\\
	&&3.\forall n_1,va_5((n_1+1)) = ite((vb_5(n_1)<m),\\ &&\hspace*{5em}(va_5(n_1)-1),va_5(n_1))\\ 
	&&3.\forall n_1,vb_5((n_1+1)) = ite((vb_5(n_1)<m),\\ &&\hspace*{5em}(vb_5(n_1)+1),0)\\
	&&6.va_5(0) = n, vb_5(0) = 0\\
	&&7.\neg (va_5(N_1)>0)\\
	&&8.\forall n_1.(n_1<N_1) \rightarrow (va_5(n_1)>0)
\end{eqnarray*}

Then RS find the closed form solution of the conditional recurrences of $va_5$ and $vb_5$
\begin{eqnarray*}
	&&\hspace*{-5em}va_5(n_1)=ite(0<n_1\leq C_1,n-n_1,0)\\
	&&\hspace*{-5em}vb_5(n_1)=ite(0<n_1\leq C_1,n_1,C_1)
\end{eqnarray*}

We can derived $C_1=m$ by solving following additional axoims
\begin{eqnarray*}
	&&\hspace*{-5em}\forall n_1. 0 \leq n_1<C_1 \rightarrow n_1<m \\
	&&\hspace*{-5em}\neg (C_1 < m)
\end{eqnarray*}

After subsituting $va_5(n_1)$ and $vb_5(n_1)$ and getting rid of $va_5(n_1+1)$ and $vb_5(n_1+1)$ results the following equations:

\begin{eqnarray*}
	&&1. n_1 = n, m_1 = m\\
	&&2. va_1 = ite(0<N_1\leq m,n-N_1,0) \\
	&&3. vb_1 = ite(0<N_1\leq m,N_1,m)\\
	&&4.\neg (ite(0<N_1\leq m,n-N_1,0)>0)\\
	&&5.\forall n_1.(n_1<N_1) \rightarrow (ite(0<n_1\leq m,n-n_1,0)>0)
\end{eqnarray*}

Now we can derived that $\mathcal{B}^{\vec{X}}=n$ from equation (4) and (5).

\subsection{SPEED 3}

The following example is taken from benchmark of SPEED~\cite{speed1}.
\begin{verbatim}
1. int n, m, i = n;
2. while (i > 0) {
3.     if (i < m) { 
4.         i = i - 1;
5.     } else {
6.         i = i - m;
7. }
8.}
\end{verbatim}




Our translator would be translated to a set of axioms $\Pi_P^{\vec{X}}$ like the following:
\begin{eqnarray*}
	&&1. n_1 = n, m_1 = m, i_1 = i_3(N_1)\\
	&&2.\forall n_1,i_3((n_1+1)) = ite((i_3(n_1)<m),\\ &&\hspace*{5em}(i_3(n_1)-1),(i_3(n_1)-m))\\ 
	&&3.i_3(0) = n\\
	&&4.\neg (i_3(N_1)>0)\\
	&&5.\forall n_1.(n_1<N_1) \rightarrow (i_3(n_1)>0)
\end{eqnarray*}

Then RS find the closed form solution of the conditional recurrence of $i_3$ using case analysis as follows

\subsubsection{Case 1} When $i_3(0)<m$ holdes, RS will find the following closed form solution
\begin{eqnarray*}
	&&\hspace*{-5em}i_3(n_1)=n-n_1
\end{eqnarray*}

After subsituting $i_3(n_1)$ and getting rid of $i_3(n_1+1)$ results the following equations:

\begin{eqnarray*}
	&&1. n_1 = n, m_1 = m, i_1 = n-N_1\\
	&&2.\neg (n-N_1>0)\\
	&&3.\forall n_1.(n_1<N_1) \rightarrow (n-n_1>0)
\end{eqnarray*}

Now we can derived that $\mathcal{B}^{\vec{X}}_{case1}=n$ from equation (2) and (3).
\subsubsection{Case 2} When $neg (i_3(0)<m)$ holdes, RS will find the following closed form solution
\begin{eqnarray*}
	&&\hspace*{-5em}i_3(n_1)=n-m*n_1
\end{eqnarray*}

After subsituting $i_3(n_1)$ and getting rid of $i_3(n_1+1)$ results the following equations:

\begin{eqnarray*}
	&&1. n_1 = n, m_1 = m, i_1 = n-m*N_1\\
	&&2.\neg (n-m*N_1>0)\\
	&&3.\forall n_1.(n_1<N_1) \rightarrow (n-m*n_1>0)
\end{eqnarray*}

Now we can derived that $\mathcal{B}^{\vec{X}}_{case2}=n/m$ from equation (2) and (3).

We can derived that $\mathcal{B}^{\vec{X}}=max(n,n/m)=n$ .

\subsection{SPEED 4}

The following example is taken from benchmark of SPEED~\cite{speed1}.
\begin{verbatim}
1.  int n, m, dir, i;
2.  assume(0<m);
3.  assume(m<n);
4.  i = m;
5.  while (0 < i && i < n) {
6.       if (dir == 1) i++;
7.       else i--;
8.  }
\end{verbatim}




Our translator would be translated to a set of axioms $\Pi_P^{\vec{X}}$ like the following:
\begin{eqnarray*}
	&&1. m1 = m, dir1 = dir, n1 = n, i_1 = i_2(N_1)\\
	&&2.\forall n_1.i_2((n_1+1)) = ite((dir==1),\\ &&\hspace*{5em}(i_2(n_1)+1),i_2(n_1)-1)\\ 
	&&3.i_2(0) = m\\
	&&4.\neg (0<i_2(N_1)) \land (i_2(N_1)<n)\\
	&&5.\forall n_1.(n_1<N_1) \rightarrow ((0<i_2(n_1)) \land (i_2(n_1)<n))
\end{eqnarray*}



Then RS find the closed form solution of the conditional recurrence of $i_3$ using case analysis under the assumptions $0<m$ and $m<n$ as follows

\subsubsection{Case 1} When $dir==1$ holdes, RS will find the following closed form solution
\begin{eqnarray*}
	&&\hspace*{-5em}i_2(n_1)=m+n_1
\end{eqnarray*}

After subsituting $i_3(n_1)$ and getting rid of $i_3(n_1+1)$ results the following equations:

\begin{eqnarray*}
	&&1. n_1 = n, m_1 = m, i_1 = m+N_1\\
	&&2.\neg (0<m+N_1) \land (m+N_1<n)\\
    &&3.\forall n_1.(n_1<N_1) \rightarrow ((0<m+n_1) \land (m+n_1<n))
\end{eqnarray*}

Now we can derived that $\mathcal{B}^{\vec{X}}_{case1}=n-m$ from equation (2) and (3).
\subsubsection{Case 2} When $neg (dir==1)$ holdes, RS will find the following closed form solution
\begin{eqnarray*}
	&&\hspace*{-5em}i_3(n_1)=m-n_1
\end{eqnarray*}

After subsituting $i_3(n_1)$ and getting rid of $i_3(n_1+1)$ results the following equations:

\begin{eqnarray*}
	&&1. n_1 = n, m_1 = m, i_1 = m-N_1\\
	&&2.\neg (0<m-N_1) \land (m-N_1<n)\\
	&&3.\forall n_1.(n_1<N_1) \rightarrow ((0<m-n_1) \land (m-n_1<n))
\end{eqnarray*}

Now we can derived that $\mathcal{B}^{\vec{X}}_{case2}=m$ from equation (2) and (3).

We can derived that $\mathcal{B}^{\vec{X}}=max(m,n-m)$ .

\subsection{SPEED 5}

The following example is taken from benchmark of SPEED~\cite{speed1}.
\begin{verbatim}
1.  int n, i=0, j;
2.  while (i < n)
3.  {
4.     j = i + 1;
5.     while (j < n)
6.     {
7.         if (nondet_int() > 0){
8.              j = j - 1; n = n - 1;
9.          }
10.          j = j + 1;
11.     }
12.     i = i + 1;
13.  }
\end{verbatim}







Our translator would be translated to a set of axioms $\Pi_P^{\vec{X}}$ like the following:
\begin{eqnarray*}
	&&1.i_1 = (N_2+0), j_1 = j_7(N_2), n_1 = n_7(N_2)\\
	&&2.\forall n_1,n_2.j_4((n_1+1),n_2) = (ite((nondet\_int_2(n_1,n_2)>0),\\ &&\hspace*{5em}(j_4(n_1,n_2)-1),j_4(n_1,n_2))+1)\\ 
	&&3.\forall n_1,n_2.n_4((n_1+1),n_2) = ite((nondet\_int_2(n_1,n_2)>0),\\ &&\hspace*{5em}(n_4(n_1,n_2)-1),n_4(n_1,n_2))\\ 
	&&4.\forall n_2.j_4(0,n_2) = ((n_2+0)+1)\\
	&&5.\forall n_2.n_4(0,n_2) = n_7(n_2)\\
	&&6.\forall n_2.\neg (j_4(N_1(n_2),n_2)<n_4(N_1(n_2),n_2))\\
	&&7.\forall n_1,n_2.(n_1<N_1(n_2)) \rightarrow (j_4(n_1,n_2)<n_4(n_1,n_2))\\
	&&8.\forall n_2.j_7((n_2+1)) = j_4(N_1(n_2),n_2)\\
	&&9.\forall n_2.n_7((n_2+1)) = n_4(N_1(n_2),n_2)\\
	&&10.j_7(0) = j, n_7(0) = n\\
	&&11.\neg ((N_2+0)<n_7(N_2))\\
	&&12.\forall n_2.(n_2<N_2) \rightarrow ((n_2+0)<n_7(n_2))
\end{eqnarray*}

Then RS find the closed form solution of the conditional recurrence of $j_4$ and $n_4$, which is associated  with inner loop, using case analysis as follows
\subsubsection{Case 1} When $\forall n_1,n_2.(nondet\_int_2(n_1,n_2)>0)$ holdes, RS will find the following closed form solution

\begin{eqnarray*}
	&&\forall n_2.j_4(n_1,n_2) = ((n_2+0)+1)\\  
	&&\forall n_2.n_4(n_1,n_2) = n_7(n_2)-n_1 
\end{eqnarray*}

After subsituting $j_4(n_1,n_2)$,$n_4(n_1,n_2)$ and getting rid of $j_4(n_1+1,n_2)$,$n_4(n_1+1,n_2)$ results the following equations:

\begin{eqnarray*}
	&&1.i_1 = (N_2+0), j_1 = j_7(N_2), n_1 = n_7(N_2)\\
	&&2.\forall n_2.\neg (N_1(n_2)<n_7(n_2)-N_1(n_2))\\
	&&3.\forall n_1,n_2.(n_1<N_1(n_2)) \rightarrow (n_1<n_7(n_2)-n_1)\\
	&&4.\forall n_2.j_7((n_2+1)) = ((n_2+0)+1)\\
	&&5.\forall n_2.n_7((n_2+1)) = n_7(n_2)-N_1(n_2)\\
	&&6.j_7(0) = j, n_7(0) = n\\
	&&7.\neg ((N_2+0)<n_7(N_2))\\
	&&8.\forall n_2.(n_2<N_2) \rightarrow ((n_2+0)<n_7(n_2))
\end{eqnarray*}



Then RS find the closed form solution of the conditional recurrence of $j_4$ and $n_4$, which is associated  with inner loop, using case analysis as follows

\[n_7(n_2) = n-\sum_{i=1}^{n_2}N_1(i)\]

After subsituting $n_7(n_2)$ and getting rid of $n_7(n_2+1)$ results the following equations:

\begin{eqnarray*}
	&&1.i_1 = (N_2+0), j_1 = ite(N_2==0,j,(N_2)\\
	&&2.n_1 = n-\sum_{i=1}^{N_2}N_1(i)\\
	&&3.\forall n_2.\neg (N_1(n_2)<n-\sum_{i=1}^{n_2}N_1(i)-N_1(n_2))\\
    &&4.\forall n_1,n_2.(n_1<N_1(n_2)) \rightarrow (n_1<n-\sum_{i=1}^{n_2}N_1(i)-n_1)\\
	&&5.\neg ((N_2+0)<n-\sum_{i=1}^{N_2}N_1(i))\\
	&&6.\forall n_2.(n_2<N_2) \rightarrow ((n_2+0)<n-\sum_{i=1}^{n_2}N_1(i))
\end{eqnarray*}

\subsubsection{Case 2} When $\forall n_1,n_2.\neg (nondet\_int_2(n_1,n_2)>0)$ holdes, RS will find the following closed form solution

\begin{eqnarray*}
	&&\forall n_2.j_4(n_1,n_2) = ((n_2+0)+1)+n_1\\  
	&&\forall n_2.n_4(n_1,n_2) = n_7(n_2)+n_1 
\end{eqnarray*}

After subsituting $j_4(n_1,n_2)$,$n_4(n_1,n_2)$ and getting rid of $j_4(n_1+1,n_2)$,$n_4(n_1+1,n_2)$ results the following equations:

\begin{eqnarray*}
	&&1.i_1 = (N_2+0), j_1 = j_7(N_2), n_1 = n_7(N_2)\\
	&&2.\forall n_2.\neg (N_1(n_2)<n_7(n_2)+N_1(n_2))\\
	&&3.\forall n_1,n_2.(n_1<N_1(n_2)) \rightarrow (n_1<n_7(n_2)+n_1)\\
	&&4.\forall n_2.j_7((n_2+1)) = ((n_2+0)+1)-N_1(n_2)\\
	&&5.\forall n_2.n_7((n_2+1)) = n_7(n_2)+N_1(n_2)\\
	&&6.j_7(0) = j, n_7(0) = n\\
	&&7.\neg ((N_2+0)<n_7(N_2))\\
	&&8.\forall n_2.(n_2<N_2) \rightarrow ((n_2+0)<n_7(n_2))
\end{eqnarray*}



Then RS find the closed form solution of the conditional recurrence of $j_4$ and $n_4$, which is associated  with inner loop, using case analysis as follows

\[n_7(n_2) = n+\sum_{i=1}^{n_2}N_1(i)\]

After subsituting $n_7(n_2)$ and getting rid of $n_7(n_2+1)$ results the following equations:

\begin{eqnarray*}
	&&1.i_1 = (N_2+0), j_1 = ite(N_2==0,j,(N_2)\\
	&&2.n_1 = n+\sum_{i=1}^{N_2}N_1(i)\\
	&&3.\forall n_2.\neg (N_1(n_2)<n-\sum_{i=1}^{n_2}N_1(i)+N_1(n_2))\\
	&&4.\forall n_1,n_2.(n_1<N_1(n_2)) \rightarrow (n_1<n-\sum_{i=1}^{n_2}N_1(i)+n_1)\\
	&&5.\neg ((N_2+0)<n+\sum_{i=1}^{N_2}N_1(i))\\
	&&6.\forall n_2.(n_2<N_2) \rightarrow ((n_2+0)<n+\sum_{i=1}^{n_2}N_1(i))
\end{eqnarray*}
\subsection{SPEED 6}

The following example is taken from benchmark of SPEED~\cite{speed1}.
\begin{verbatim}
1.  int n;
2.  while (n>0)
3.  {
4.     n=n-1;
5.     while (n>0) 
6.     {
7.         if (__VERIFIER_nondet_int())
                  break;
8.         n=n-1;
9.     }
10.  }
\end{verbatim}

Our translator would be translated to a set of axioms $\Pi_P^{\vec{X}}$ like the following:
\begin{eqnarray*}
	&&1.break\_1\_flag_1 = break\_1\_flag_7(N_2), n_1 = n_7(N_2)\\
	&&2.\forall n_1,n_2.n_5((n_1+1),n_2) = \\ &&\hspace*{5em}ite((ite((nondet\_int_2(n_1,n_2)>0),1,0)==0),\\ 
    &&\hspace*{7em}(n_5(n_1,n_2)-1),n_5(n_1,n_2))\\ 
	&&3.\forall n_1,n_2.break\_1\_flag_5((n_1+1),n_2) = \\ &&\hspace*{5em}ite((nondet_int_2(n_1,n_2)>0),1,0)\\ 
	&&4.\forall n_2.break\_1\_flag_5(0,n_2) = break\_1\_flag_7(n_2)\\
	&&5.\forall n_2.n_5(0,n_2) = (n_7(n_2)-1)\\
	&&6.\forall n_2.\neg ((n_5(<N_1(n_2),n_2)>0)\\
	&&\hspace*{5em}\land (break\_1\_flag_5(<N_1(n_2),n_2)==0))\\
\end{eqnarray*}
\begin{eqnarray*}
	&&7.\forall n_1,n_2.(n_1<N_1(n_2)) \rightarrow ((n_5(n_1,n_2)>0)\\
	&& \hspace*{5em}\land (break\_1\_flag_5(n_1,n_2)==0))\\
	&&8.\forall n_2.break\_1\_flag_7((n_2+1)) = break\_1\_flag_5(N_1(n_2),n_2)\\
	&&9.\forall n_2.n_7((n_2+1)) = n_5(N_1(n_2),n_2)\\
	&&10.break\_1\_flag_7(0) = 0, n_7(0) = n\\
	&&11.\neg (n_7(N_2)>0)\\
	&&12.\forall n_2.(n_2<N_2) \rightarrow (n_7(n_2)>0)
\end{eqnarray*}


By analyzing the equations (6) and (7), we have found that loop terminates only when break excuted. We need to check all possible exution path of this loop end up excuting break. For that we will use case analysis to we find the closed form solution of the conditional recurrences of $n_5$ and $n_7$

\subsubsection{Case 1} When break never excute which means $\forall n_1,n_2. \neg (nondet\_int_2(n_1,n_2)>0)$, then we tried to find the closed form solution of the following recurrence

\[n_5((n_1+1),n_2) = (n_5(n_1,n_2)-1)\]

Then RS find the closed form solution of the recurrence of $n_5$, which is associated  with inner loop, using case analysis. After subsituting $n_5(n_1,n_2)=(n_7(n_2)-1)-n_1$ and getting rid of $n_5((n_1+1),n_2)$ results the following equations:

\begin{eqnarray*}
	&&1.break\_1\_flag_1 = break\_1\_flag_7(N_2), n_1 = n_7(N_2)\\
	&&2.\forall n_2.\neg (((n_7(n_2)-1)-N_1(n_2)>0) \land (0==0))\\
	&&3.\forall n_1,n_2.(n_1<N_1(n_2)) \rightarrow ((n_7(n_2)-1)-N_1(n_2)>0)\\
	&&\hspace*{5em}\land (0==0))\\
	&&4.\forall n_2.break\_1\_flag_7((n_2+1)) = 0\\
	&&5.\forall n_2.n_7((n_2+1)) = (n_7(n_2)-1)-N_1(n_2)\\
	&&6.break\_1\_flag_7(0) = 0, n_7(0) = n\\
	&&7.\neg (n_7(N_2)>0)\\
	&&8.\forall n_2.(n_2<N_2) \rightarrow (n_7(n_2)>0)
\end{eqnarray*}

Then RS find the closed form solution of the recurrence of $n_7$, which is associated  with inner loop, using case analysis. After subsituting $n_7(n_2)=n -\sum_{i=1}^{n_2}N_1(i)-n_2$ and getting rid of $n_7(n_2)$ results the following equations:

\begin{eqnarray*}
	&&1.break\_1\_flag_1 = break\_1\_flag_7(N_2), n_1 = n_7(N_2)\\
	&&2.\forall n_2.\neg (((n -\sum_{i=1}^{n_2}N_1(i)-n_2-1)\\
	&&\hspace*{5em}-N_1(n_2)>0) \land (0==0))\\
	&&3.\forall n_1,n_2.(n_1<N_1(n_2)) \rightarrow ((n -\sum_{i=1}^{n_2}N_1(i)-n_2-1)\\
	&&\hspace*{5em}-N_1(n_2)>0) \land (0==0))\\
	&&4.\neg (n -\sum_{i=1}^{N_2}N_1(i)-N_2>0)\\
	&&5.\forall n_2.(n_2<N_2) \rightarrow (n -\sum_{i=1}^{n_2}N_1(i)-n_2>0)
\end{eqnarray*}

\subsection{SPEED 7}
The following example is taken from benchmark of SPEED~\cite{speed1}.
\begin{verbatim}
1.  int x,y,n,m;
2.  while (n > x)
3.  {
4.     if (m > y) {y = y + 1;}
5.     else { x = x + 1; }
6.  }
\end{verbatim}

Our translator would be translated to a set of axioms $\Pi_P^{\vec{X}}$ like the following:
\begin{eqnarray*}
	&&1.m_1 = m, n_1 = n, y_1 = y_3(N_1), x_1 = x_3(N_1)\\
	&&2.\forall n_1.y_3((n_1+1)) = ite((m>y_3(n_1)),\\ 
	&&\hspace*{5em}(y_3(n_1)+1),y_3(n_1))\\ 
	&&3.\forall n_1.x_3((n_1+1)) = ite((m>y_3(n_1)), \\ &&\hspace*{5em}x_3(n_1),(x_3(n_1)+1))\\ 
	&&10.y_3(0) = y, x_3(0) = x\\
	&&11.\neg (n>x_3(N_1))\\
	&&12.\forall n_1.(n_1<N_1) \rightarrow (n>x_3(n_1))
\end{eqnarray*}

Then RS find the closed form solution of the conditional recurrence of $x_3$ and $y_3$ using case analysis as follows

\subsubsection{Case 1} When $(m>y_3(0)$ holdes, RS will find the following closed form solution

\begin{eqnarray*}
	&&y_3(n_1) = ite(0\leq n_1 \leq C_1,y+n_1,y+C_1)\\ 
	&&x_3(n_1) = ite(0\leq n_1 \leq C_1,x,x+n_1)
\end{eqnarray*}

We can derived $C_1=m$ by solving following additional axoims
\begin{eqnarray*}
	&&\hspace*{-5em}\forall n_1. 0 \leq n_1<C_1 \rightarrow m>n_1 \\
	&&\hspace*{-5em}\neg (m>C_1)
\end{eqnarray*}

After subsituting $x_3(n_1)$ and $y_3(n_1)$ and getting rid of $x_3(n_1+1)$ and $y_3(n_1+1)$ results the following equations:

\begin{eqnarray*}
	&&1.m_1 = m, n_1 = n,\\
	&&2.y_1 = ite(0\leq N_1 \leq m,y+N_1,y+m)\\
	&&3.x_1 = ite(0\leq N_1 \leq m,x,x+N_1)\\
	&&4.\neg (n>ite(0\leq n_1 \leq m,x,x+N_1))\\
	&&5.\forall n_1.(n_1<N_1) \rightarrow (n>ite(0\leq n_1 \leq m,x,x+n_1))
\end{eqnarray*}

Now we can derived that $\mathcal{B}^{\vec{X}}_{Case_1}=n-x$ from equation (4) and (5).

\subsubsection{Case 2} When $\neg (m>y_3(0)$ holdes, RS will find the following closed form solution

\begin{eqnarray*}
	&&y_3(n_1) = y\\ 
	&&x_3(n_1) = x+n_1
\end{eqnarray*}



After subsituting $x_3(n_1)$ and $y_3(n_1)$ and getting rid of $x_3(n_1+1)$ and $y_3(n_1+1)$ results the following equations:

\begin{eqnarray*}
	&&1.m_1 = m, n_1 = n,\\
	&&2.y_1 = y\\
	&&3.x_1 = x+N_1\\
	&&4.\neg (n>x+N_1)\\
	&&5.\forall n_1.(n_1<N_1) \rightarrow n>x+n_1
\end{eqnarray*}

Now we can derived that $\mathcal{B}^{\vec{X}}_{Case_2}=n-x$ from equation (4) and (5).


%\section{Appendix}







\end{document}
