\section{Translation}\label{sec:Translation}
Our translator consider programs in the following language:
\begin{verbatim}
E ::= array(E,...,E) |
operator(E,...,E)
B ::= E = E |
boolean-op(B,...,B)
P ::= array(E,...,E) = E | 
if B then P else P |
P; P |
while B do P 
\end{verbatim}

%Our system supports the full C99 language (according to the standard ISO/IEC 9899).
Given a program $P$, and a language $\vec{X}$, our system generates a set of first-order axioms denoted by $\Pi_P^{\vec{X}}$ that captures the changes of $P$ on $\vec{X}$. Here, a language means a set of functions and predicate symbols. For $\Pi_P^{\vec{X}}$ to be correct, $\vec{X}$ needs to include all program variables in $P$ as well as any functions and predicates that can be changed by $P$. The axioms in the set $\Pi_P^{\vec{X}}$ are generated inductively on the structure of $P$. The algorithm is described in detail in \cite{Lin20161}
and implementation is explained in 
\cite{DBLP:conf/synasc/pritom17}. The inductive cases of translations are given in the table provided in the supplementary information\footnote{\url{https://github.com/VerifierIntegerAssignment/VIAP_ARRAY/blob/master/Document/Inductive_Translation.pdf}}. We have extended our translation programs with arrays; the extension is described in detail in \cite{viap-array}.


The translation for the sequence, conditonal, and while loops remain the same as that of the deterministic programs presented in \cite{Lin20161}.  We have updated those rules by integrating the simplification process, which are briefly described as follows:


\begin{defn} \label{Rule3} When \verb-P- is \verb-if B then P1 else P2- then $\Pi_P^{\vec{X}}$ is the set of following axioms:
	
\vspace{2mm}
$B \rightarrow \varphi$, for each $\varphi \in \Pi_{P1}^{\vec{X}},$

\vspace{2mm}
$\neg B \rightarrow \varphi$, for each $\varphi \in \Pi_{P1}^{\vec{X}},$
\vspace{2mm}

We assume here that for each boolean expression $B$, there is a corresponding formula $B$ in our first-order language.
	

\end{defn}


Translator uses a systematic technique for the simplifying condition of each axoim $\varphi$ $\in$ $\Pi_P^{\vec{X}}$ of the following form where $\Pi_P^{\vec{X}}$ is the set of axioms generated after the translation of the conditional statement:
\begin{eqnarray}
\forall \vec{x}.X_1(\vec{x}) &=& ite(B,\hat{E_{1}},\hat{E}_{2})\label{rec2}
\end{eqnarray}

where $E_{1}$, $E_{2}$ are the translation of the expressions. It uses two different rules which are presented as follows:
\begin{itemize}
	\item When $B$ is boolean expression, and $E_{1}=E_{2}$, then $\varphi$ is simplified to $X_1(\vec{x})=E_{1}$.
	\item When $B$ is boolean expression, and evaluate $B$ is true, then $\varphi$ is simplified to $X_1(\vec{x})=E_{1}$.
\end{itemize}

\begin{defn}\label{Rule4} When \verb-P- is \verb-while B do P1- then $\Pi_P^{\vec{X}}$ is the set of following axioms:
	\vspace{2mm}
	$\varphi(n)$, for each $\varphi \in \Pi_{P1}^{\vec{X}}$,
	\vspace{2mm}
	
	$X^i(\vec{x})$ = $X^i(\vec{x}, 0)$, for each $X^i \in  \vec{X}$
	
	\vspace{2mm}
	$smallest(N, n, \neg B(n)),$
	
	\vspace{2mm}
	$X^i_1(\vec{x})$ = $X^i(\vec{x}, N)$, for each $X^i \in X$
	\vspace{2mm}
	where n is a new natural number variable not already in $\varphi$,
	and $N$ a new constant not already used in $\Pi_{P1}^{\vec{X}}$. 
	For each formula or term $\alpha$, $\alpha(n)$ is defined inductively as
	follows: it is obtained from $\alpha$ by performing the following
	recursive substitutions:
	\begin{itemize}
		\item for each $X^i 
		\in \vec{X}$ , replace all occurrences of 
		$X^{i}_1(e_1, ..., e_k)$ by $X^i(e_1(n), ..., e_k(n)$$, n + 1)$, and
		\item for each program variable $X$ in $\alpha$, replace all occurrences
		of $X(e_1, ..., e_k)$ by $X(e_1(n), ..., e_k(n), n)$. Notice
		that this replacement is for every program variable
		$X$, including those not in $\vec{X}$ .
	\end{itemize}
	$smallest(N, n, \neg B(n))$ is a shorthand for the following formula
	\begin{eqnarray}
	&& \neg  B(N), \label{smallest1}\\
	&&\forall n. n< N\rightarrow B(n) \label{smallest2}
	\end{eqnarray}
\end{defn}


Lets consider $\vec{\sigma}$ represents the set of inductively defined axioms during the translation. If all the condition axioms $\vec{\sigma}$ are of the following form for some $h\geq1$
\begin{eqnarray}
\hspace*{1em}X^i(e_1(n), ..., e_k(n),n+1) &=& \nonumber\\
&&\hspace*{-12em}ite(\theta_1,f_1(X^i(e_1(n), ..., e_k(n)),n),\nonumber\\
&&\hspace*{-10em}ite(\theta_2,f_2(X^i(e_1(n), ...,e_k(n)),n)...,\nonumber\\ &&\hspace*{-8em}ite(\theta_h,f_h(X^i(e_1(n), ..., e_k(n)),n)\nonumber\\
&&\hspace*{-6em},f_{h+1}(X^i(e_1(n), ..., e_k(n)),n))), \nonumber\\
&&\hspace*{-4em} \mbox{ for }1\leq i\leq h \land X^i \in \vec{X}, \label{rec3}
\end{eqnarray}

where $\theta_1,\theta_2,...,\theta_{h}$ are boolean expressions, and 
$f_1(x,y),f_2(x,y),....,f_{h+1}(x,y)$ are polynomial functions of $x$ and $y$. The translator uses a systematic technique to reconstruct $\Pi_P^{\vec{X}}$ as following:

\begin{itemize}
	\item if all the conditions $\theta_i$, $1\leq i\leq h$,
	in it are independent of the recurrence variable $n$, the translator splits the set of axioms $\Pi_P^{\vec{X}}$ to a sequence of set of axioms $\Pi_{P^{1}}^{\vec{X}}$, $\Pi_{P^{2}}^{\vec{X}}$,...,$\Pi_{P^{h}}^{\vec{X}}$, $\Pi_{P^{h+1}}^{\vec{X}}$ corresponding to conditions $\theta_1, \theta_2,...,\theta_h, \theta_{h+1}$ respectively where $\theta_{h+1}   = (\neg \theta_1 \land \neg \theta_2 \land..... \land \neg \theta_h)$.
	For any $\Pi_i$ such that $1\leq i\leq h$. $\Pi_{P^{1}}^{\vec{X}}$ consist of the following set of axioms along with all non-conditional inductively defined axioms of $\Pi_P^{\vec{X}}$  which corresponds to boolean expression $\theta_i$
	
    \vspace{2mm}
	$X^i(\vec{x},0) = X^i(\vec{x})$,  for each $X^i \in  \vec{X}$

	\vspace{2mm}
	
	$smallest(N, n, \neg (B(n) \land \theta_i))$, 
	
	\vspace{2mm}
	$X^i(e_1(n), ..., e_k(n),n+1) = f_i(X^i(e_1(n), ...,e_k(n)),n)$,

	\vspace{2mm} 
	$X^i_1(\vec{x}) = X^i(\vec{x}, N), \text{ for each $X^i$ $\in X$ }$

	\vspace{2mm}

     Then reconstruct $\Pi_P^{\vec{X}}$ from $\Pi_{P^{1}}^{\vec{X}}$, $\Pi_{P^{2}}^{\vec{X}}$,...,$\Pi_{P^{h}}^{\vec{X}}$, ,$\Pi_{P^{h+1}}^{\vec{X}}$ following the translation rule definition presented in~\ref{Rule2}.
     
	\vspace{2mm}
	$\varphi(\vec{X_1}/\vec{Y_1}) \rightarrow \varphi$, for each $\varphi \in \Pi_{P^{1}}^{\vec{X}},$

	\vspace{2mm}
	$\varphi(\vec{X}/\vec{Y_1}) \rightarrow \varphi$, for each $\varphi \in \Pi_{P^{2}}^{\vec{X}}$
	\vspace{2mm}
	
    $\vdots$
    
	\vspace{2mm}
	$\varphi(\vec{X_1}/\vec{Y_h}) \rightarrow \varphi$, for each $\varphi \in \Pi_{P^{h}}^{\vec{X}},$

	\vspace{2mm}
	$\varphi(\vec{X}/\vec{Y_h}) \rightarrow \varphi$, for each $\varphi \in \Pi_{P^{h+1}}^{\vec{X}}$
	\vspace{2mm}
	
	where any $\vec{Y_i} = (Y^1_{i}, ..., Y^k_{i})$ is a tuple of new function symbols such that each $Y^j_{i}$ is of the same arity as $X^j$
in $\vec{X}$ such that $1\leq j \leq k$ , $\varphi(\vec{X_1}/\vec{Y_i} )$ is the result of replacing in $\varphi$
each occurrence of $X^j_1$ by $Y^j_{i}$, and similarly for $\varphi(X/\vec{Y_i})$.

 \item if the condition $\theta_i$, $1\leq i\leq h$, contains only
 arithmetic comparisons $>$ or $<$, polynomial functions of $n$, and does
 not mention $X(n)$. Then translator tries to compute the ranges
 of $\theta_i$ as follows. For each $1\leq i\leq h$, let
 \[
 \phi_i = \neg \theta_1\land\cdots\land\neg \theta_{i-1}\land \theta_i.
 \]
 Thus $\phi_i$ is the condition
 for $X(n+1)$ to take the value $f_i(X(n),n)$. Our system then tries
 to compute $h$ constants $C_1,...C_h$ such that
 \[
 0=C_0<C_1<\cdots<C_h,
 \]
 and for each $i\leq h+1$,
 \vspace{-0.2cm}
 \begin{eqnarray*}
 	&&
 	\forall n. C_{i-1}\leq n< C_i\rightarrow \phi_{\pi(i)}(n) , \\
 	&&\hspace{20mm}\forall n. n\geq C_i\rightarrow \neg \phi_{\pi(i)}(n).
 \end{eqnarray*}
 where $\pi$ is a permutation on $\{1,...,n\}$. The computation of
 these constants and the associated permutation $\pi$ is done using
 an SMT solver: it first computes $k$ for which $\phi_k(0)$ holds, and then
 asks the SMT solver to find a model of
 \begin{eqnarray*}
 	&&
 	\forall n. 0\leq n< C\rightarrow \phi_k(n), \forall n. n\geq C\rightarrow \neg \phi_k(n).
 \end{eqnarray*}
 If it returns a model, then set $C_1=C$ in the model, and let $\pi(1)=k$,
 and the process continues until all $C_i$ are computed. If this fails
 at any point, then the translator aborts this simplification step and return $\Pi_P^{\vec{X}}$ .

Once the system succeeds in computing these constants, the translator reconstruct $\Pi_P^{\vec{X}}$ to a sequence of set of axioms $\Pi_{P^{1}}^{\vec{X}}$, $\Pi_{P^{2}}^{\vec{X}}$,...,$\Pi_{P^{h}}^{\vec{X}}$, ,$\Pi_{P^{h+1}}^{\vec{X}}$ corresponding to constants $C_1, C_2,...,C_h, C_{h+1}$ respectively.
For any $\Pi_i$ such that $1\leq i\leq h$. $\Pi_{P^{1}}^{\vec{X}}$ consist of the following set of axioms along with all non-conditional inductively defined axioms of $\Pi_P^{\vec{X}}$  which corresponds to constant $C_i$

    \vspace{2mm}
$X^i(\vec{x},0) = X^i(\vec{x})$,  for each $X^i \in  \vec{X}$

\vspace{2mm}

$smallest(N, n, \neg (B(n) \land n<C_i))$, 

\vspace{2mm}
$X^i(e_1(n), ..., e_k(n),n+1) = f_i(X^i(e_1(n), ...,e_k(n)),n)$,

\vspace{2mm} 
$X^i_1(\vec{x}) = X^i(\vec{x}, N), \text{ for each $X^i$ $\in X$ }$

\vspace{2mm}

Then reconstruct $\Pi_P^{\vec{X}}$ from $\Pi_{P^{1}}^{\vec{X}}$, $\Pi_{P^{2}}^{\vec{X}}$,...,$\Pi_{P^{h}}^{\vec{X}}$, ,$\Pi_{P^{h+1}}^{\vec{X}}$ following the translation rule definition presented in~\ref{Rule2} as presented in previous case.


	
\end{itemize}




